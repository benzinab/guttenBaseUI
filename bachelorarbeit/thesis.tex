%!TEX TS-program = xelatex
%!TEX encoding = UTF-8 Unicode
\documentclass[
%12pt, % Schriftgröße
%DIV10, % Änderung der Größe des Satzspiegels (bedruckbarer Bereich einer Seite), nur in Verbindung mit koma-script verwendbar
ngerman, % für Umlaute, Silbentrennung etc.
%a4paper, % Papierformat
%oneside, % einseitiges Dokument
%titlepage, % es wird eine Titelseite verwendet
%parskip=half, % Abstand zwischen Absätzen (halbe Zeile)
%headings=normal, % Größe der Überschriften verkleinern
%listof=totoc, % Verzeichnisse im Inhaltsverzeichnis aufführen
%bibliography=totoc, % Literaturverzeichnis im Inhaltsverzeichnis aufführen
%index=totoc, % Index im Inhaltsverzeichnis aufführen
%captions=tableheading, % Beschriftung von Tabellen unterhalb ausgeben
%final
]{thesis}

\usepackage[chapter]{tocbibind}

\usepackage{blindtext}
\usepackage{float}

\usepackage{glossaries}
\usepackage{biblatex}

%\setStyleFile{glossary}
\makeglossaries
% \newglossaryentry{label}{
%   name={name},
%   description={long description}
% }


% \newacronym{label}{kurz}{lang}
% Zugriff via \gls{label} und Co.
% 
% Beispiel:
% 
%   \newacronym{acr:da}{DA}{Diplomarbeit}
% 
% wird im Text
% 
%   Heute schreibe ich meine \gls{arc:da}. Diese \gls{arc:da}.
% 
% zu
% 
%   Heute schreibe ich meine Diplomarbeit (DA). Diese DA.
% 
% Nur bei der ersten Verwendung der Abkürzung wird diese in langer Form
% dargestellt. Weitere Vorkommen nutzen die Kurz-Version.
% \dfrac{num}{den}
% Neben \gls{} gibt es auch noch \Gls{} und \GLS{}, welche die Groß- und
% Klein-Schreibung beeinflussen. Beispiel:
% 
%   \newacronym{acr:http}{http}{Hypertext Transfer Protocol}
%   \gls{acr:http} --> http
%   \Gls{acr:http} --> Http
%   \GLS{acr:http} --> HTTP
% 
% Zur Vereinfachung gibt es auch die Befehle \acr{}, \Acr{} und \ACR{}. Diese
% Befehle sind äquivalent:
% 
%   \gls{acr:foo} <=> \acr{foo}
%   \Gls{acr:foo} <=> \Acr{foo}
%   \GLS{acr:foo} <=> \ACR{foo}
%


\bibliography{bib.bib}
\makeatletter

%%% USER SETTINGS

\author{Sirajeddine Ben Zinab}
\matriculationnumber{3094966}

\title{Entwicklung einer Anwendungsoberfläche für Datenbankmigration mit GuttenBase}
\subtitle{}

\date{28.\,Februar 2021}

\institute{Universität Bremen}
\department{Fachbereich 3: Mathematik und Informatik}

\examiner{Prof. Dr. Sebastian Maneth}
\supervisor{Prof. Dr. Martin Gogolla}


%%% DOCUMENT SETTINGS

% uncomment for 1.5-linespacing
\onehalfspacing

\makeatother
\endinput
%% anything below this line is ignored

\setcounter{secnumdepth}{4}

\begin{document}
% Titel
%\maketitle
% Impressumsseite
\clearpage
\thispagestyle{empty}\small
\null\vfill

\makeatletter
  \textbf{\@author}\par
  \@title\par
  \ifthesis@hassubtitle\@subtitleshort\fi

  %\dfrac{diploma}{den}name, \@department\par
  \@institute, \monthyear
\makeatother

\normalsize

\clearpage
\section*{Selbstständigkeitserklärung}

Hiermit erkläre ich, dass ich die vorliegende Arbeit selbstständig angefertigt,
nicht anderweitig zu Prüfungszwecken vorgelegt und keine anderen als die
angegebenen Hilfsmittel verwendet habe. Sämtliche wissentlich verwendete
Textausschnitte, Zitate oder Inhalte anderer Verfasser wurden ausdrücklich als
solche gekennzeichnet.

Bremen, den \makeatletter\@date\makeatother

\vspace*{1em}
\rule{15em}{0.16667pt}\\
\makeatletter\@author\makeatother

%
%\clearpage
%\section*{Danksagung}
An dieser Stelle möchte ich allen Menschen meinen Dank aussprechen, die mir geholfen und mich dabei unterstützt haben, diese Arbeit zu erstellen. \\
Als erstes möchte ich mich bei meinem Betreuer Prof. Dr. Sebastian Maneth bedanken, der mich während meiner Bachelorarbeit unterstützt und mir wertvolle Tipps gegeben hat.\\
Ich bedanke mich außerdem bei Herrn Kolja Koischwitz und Herrn Markus Dahm von der Firma akquinet AG für die Kooperation. Die freundliche und zuverlässige Zusammenarbeit hat mir eine praxisnahe Forschung ermöglicht und wertvolle Einblicke gewährt.\\
Letztlich richte ich auch ein Dankeschön an Herr Ahmed Azzouz für das Korrekturlesen meiner Arbeit sowie an meine Eltern und Freunde, die mir während meines ganzen Studium unterstützt haben.

\clearpage
\section*{Zusammenfassung}

Migration ist in der Wissenschaft kein neues Thema. Es existieren zahlreiche Methoden und Frameworks zur Beschreibung, Analyse und Implementierung der Migration. Dies gilt auch für Datenbankverwaltungssysteme (DBMS). \\ 
In dieser Arbeit werden aktuelle Tools für die Datenbankmigration vorgestellt. Dabei werden bedeutsamen Eigenschaften der Open-Source-Bibliothek GuttenBase erläutert.\\
Außerdem hat diese Arbeit hauptsächlich den Entwurf, die Implementierung und die Evaluation eines Tools für Datenbank Migration zwischen verschiedenen Datenbanksystemen (DBMS) basierend auf GuttenBase zum Thema.
Um die Nutzung der GuttenBase-Bibliothek für möglichst viele Nutzer zu ermöglichen, erfolgt die Umsetzung als ein IntelliJ Plugin (IDEA).\\ 
Diese Bachelorarbeit wurde bei der Firma Akquinet AG in Bremen im Zeitraum von September 2020 bis zum März 2021 erstellt und stellt den Abschluss meines Bachelorstudiums an der Universität Bremen dar. \\ 
Der Text liegt in deutscher Sprache vor.







\frontmatter
\phantomsection
%\addcontentsline{toc}{section}{\contentsname}
\tableofcontents




\cleardoublepage
\listoffigures

\cleardoublepage
\listoftables

%\clearpage
%\listoftodos

\cleardoublepage
\mainmatter

\chapter{Einleitung}

\section{Problemstellung und Motivation}

%Problemstellung/Forschungsfrage: 
%Wie lässt sich ein Tool für DB Migration basierend auf GB entwickeln?
%Was sind die aktuellen Tools für Datenbank Migration?
%Wie lässt sich die GB Bibliothek optimieren?

%\section{Problemstellung }
IT-Migration ist seit Anbeginn des Informationszeitalter ein wichtiger Bestandteil der Informationsverarbeitung \cite{wachter2015systemkonsolidierung}. Wie die Harware, Betriebssysteme und Programme, werden Datenbanken auch häufig migriert. Der Grund dafür könnte z. B. eine Änderung der Unternehmensrichtlinien sein. \\ 
Bei einer Datenbank Migration werden Daten von einer Quell-Datenbank zu einer Ziel-Datenbank verschoben. \\
%Trotz der Relevanz der Datenbank Migration, ist die Entwicklung und die Forschung in diesem Bereich in den letzten Jahren sehr gering. Aus diesem Grunde stellt sich die Frage, wie sich die Datenbank Migration optimieren lässt.\\
%\section{Motivation }
Es gibt viele Tools zum Visualisieren oder Analysieren von  Datenbanken. Ebenfalls gibt es einige Programme für Datenbank Migration. Dazu gehört die GuttenBase Bibliothek. Diese bietet durch das Hinzufügen von Migrationsoperationen eine gewisse Flexibilität während des Migrationsprozesses an. Um diese Migrationsoperationen am effizientesten auszunutzen und eine schnellere und anpassbare Migration durchzuführen, lässt sich GuttenBase stark optimieren. \\Diese Arbeit beschäftigt sich mit der Frage, wie sich die von der Firma Akquinet AG entwickelte Bibliothek optimieren lässt.
%In Guttenbase muss man bei jeder Migration das ganze Mapping selber implementieren (überschreiben). \\
%Außerdem fehlt die Möglichkeit, eine Übersicht der Datenbank zu haben und Konfigurationsschritte hinzuzufügen, während man migriert.
%Aus diesen Gründen lässt sich die Nutzung der GuttenBase Bibliothek einschränken. Deswegen bietet sich die Möglichkeit, ein Tool zu realisieren, um die Nutzung der GuttenBase Bibliothek zu optimieren und eine flexible, einfache und konfigurierbare Datenbank Migration zu ermöglichen.
\section{Zielsetzung}
Die GuttenBase Bibliothek lässt sich durch unterschiedliche Weiterentwicklungen optimieren. 
Im Rahmend dieser Arbeit sollte  eine eigene Anwendungsoberfläche (GuttenBase Plugin) für Datenbank Migration basierend auf GuttenBase konzipiert, implementiert und anschließend evaluiert werden. \\ \\
Das GuttenBase Plugin soll die wichtigsten Funktionalitäten von GuttenBase unterstützen. Diese werden bei der Anforderungsanalyse genauer erläutert (siehe Abschnitt \ref{anaylse}).\\
Um ein benutzerfreundliches System zu erzielen, ist es wichtig dass die zu entwickelnde Anwendungsoberfläche den Grundsätzen der Informationsdarstellung entsprechen. Diese wurden in der Norm DIN EN ISO 9241-112 vorgestellt und beinhalten folgende Grundsätze:
\begin{itemize}
	\item Entdeckbarkeit: 
	Informationen sollen bei der Darstellung erkennbar sein und als vorhanden wahrgenommen werden.
	
	\item Ablenkungsfreiheit: 
	Erforderliche Informationen sollen wahrgenommen werden, ohne Störung von weiteren dargestellten Informationen.
	
	\item Unterscheidbarkeit: 
	Elemente oder Gruppen von elementen sollen voneinander unterschieden werden können. Die Darstellung sollte die Unterscheidung bzw. Zuordnung von Elementen und Gruppen unterstützen.
	
	\item Eindeutige Interpretierbarkeit:
	Informationen sollen verstanden werden, wie es vorgesehen ist.
	
	\item Kompaktheit:
	Nur notwendige Informationen sollen dargestellt werden.
	
	\item Konsistenz:
	Informationen mit ähnlicher Absicht söllen ähnlich dargestellt werden und Informationen mit unterschiedlicher Absicht sollen in unterschiedlicher Form dargestellt werden.
	
\end{itemize}
Die genannten Grundsätze sollen im Zusammenhang mit den Gründsätzen für die Benutzer-System-Interaktion („Dialogprinzipien“) angewendet werden. Diese beinhalten, Nach der Norm DIN EN ISO 9241-11, folgende Grundsätze:
\begin{itemize}
	\item Aufgabenangemessenheit:
	Schritte sollen nicht überflüssig sein und keine irreführende Informationen beinhalten.
	
	\item Selbstbeschreibungsfähigkeit:
	Es sollen nur genau die Informationen dargestellt werden, die für einen bestimmten Schritt erforderlich sind.
	
	\item Erwartungskonformität:
	Das System verhält sich nach Durchführung einer bestimmten Aufgabe wie erwartet.
	
	\item Lernförderlichkeit:
	der Benutzer kann den entsprechenden Schritt durchführen, eine ein Vorwissen bzw. eine Schulung zu haben.
	
	\item Steuerbarkeit:
	Der Benutzer kann konsequent und ohne Umwege in Richtungen gehen, die für die zu erledigende Aufgabe erforderlich sind.
	
	\item Fehlertoleranz:
	Das System soll den Benutzer vor Fehlern schützen, und wenn Fehler gemacht werden, sollen diese mit minimalen Aufwand behoben werden können.

	\item Individualisierbarkeit:
	Der Benutzer kann Anwendungsoberfläche durch individuelle Voreinstellugen anpassen.
	
\end{itemize}
Die oben genannten Grundsätze stellen sicher, dass das GuttenBase Plugin effektiv, effizient und zufriedenstellend ist. Diese sind die drei Ziele der Gebrauchstauglichkeit (Usability).

\section{Aufbau der Arbeit}
Zu Beginn der Arbeit werden einige Grundbegriffe für Datenbank Migration erläutert. Außerdem werden die Eigenschaften der GuttenBase Bibliothek vorgestellt.\\
Zusätzlich werden aktuelle Tools für Datenbank Migration erwähnt.\\
Der Hauptteil dieser Arbeit beschäftigt sich hauptsächlich mit der Umsetzung des Guttenbase Plugins. Dabei wird zuerst eine Anforderungsanlyse durchgeführt, um den Soll-Zustand zu definieren. Um die technische Machbarkeit zu prüfen und Zeit bei der Entwicklung zu sparen, werden bei der Analyse einige GUI-Prototypen erstell. Somit wird am Anfang der Umsetzung klar sein, wie die zu entwickelnde Anwendungsoberfläche die Funktionalitäten von Guttenbase unterstützen würde. Außerdem wird die Umsetzungsform begründet. \\
Die Software Architektur erfolgt im darauffolgenden Abschnitt. Diese wird basierend auf den Siemens Blickwinkel erstellt. Zunächst werden die verwendeten Technologien sowie das Ergebnis vorgestellt.\\
Im darauffolgenden Kapitel wird das Ergebnis kurz evaluiert. Dabei wird ein Experten-Interview durchgeführt.\\
Anschließend gibt es eine Zusammenfassung sowie Ideen für Optimierungsmöglichkeiten.
\chapter{Grundlagen}
Dieses Kapitel liefert einen allgemeinen Einblick einige Grundaspekte der Datenbank Migration sowie der GuttenBase Bibliothek. Außerdem werden verwandte Arbeiten vorgestellt. 
\section{Datenbanken}
%- allgemeine Def
%- Arten von Datenabnken
%https://books.google.de/books?hl=de&lr=&id=_3XVAwAAQBAJ&oi=fnd&pg=PA14&dq=Datenbank+Grundlagen+&ots=ntu4qhQNQD&sig=LpAgd_s5f04ulLTfOYe6lEt1_Zc#v=onepage&q=Datenbank%20Grundlagen&f=false
Datenbanken spielen seit der Neuerung des IT-Zeitalter eine wichtige Rolle in dem elektronischen Datenmanagement.\\
Eine Datenbank ist eine geordnete, selstbeschreibende Sammlung von Daten, die miteinander in Beziehung stehen.
Vielmehr ist eine Datenbank ein verteiltes, integriertes Computersystem, das Nutzdaten und Metadaten enthält. Nutzdaten sind dabei die Daten, die Benutzer in der Datenbank anlegen und aus denen die Informationen gewonnen werden. Metadaten werden of auch als Daten über Daten bezeichnet und helfen, die Nutzdaten der Datenbank zu strukturieren. 
	
	
	
%\section{Datenbank Management System (DBMS)}
%- Def
%- Beispiele
Damit Datenbanken auf einem Computer verwaltet werden können, werden Datenbankmanagement Systeme (DBMS) benötigt. Diese sind leistungsfähige Programme für die flexible Speicherung und Abfrage strukturierter Daten. \\
Außerdem hilft ein DBMS bei der Organisation und Integrität von Daten und regelt den Zugruff auf Datengruppen. \\
Ein DBMS kann aus einem einzelenen Programm bestehen. Dies ist z. B. bei einem Desktop-DBMS zu sehen. Es kann jedoch aus verschiedenen Programmen bestehen, die zusammenarbeiten und die Funktion des DBMS bereitstellen. Dies ist z. B. bei den servergestützen Datenbanksystemen der Fall.\\
Um eine Datenbank Anwendung zu implementieren, sollte auf das Datenbankmodell geachtet werden. Dies stellt die Daten einer Datenbank und deren Beziehungen abstrakt dar. Meistens wird ein relationales Datenbankmodell eingesetzt. Dies hat, im Gegensatz zu den anderen Datenbankmodellen, keine strukturelle Abhängigkeit und versteckt die physikalische Komplexität der Datenbank komplett vor den Anwendern.\\
Es stehen zahlreiche Datenbankmanagementsysteme zur Verfügung. Folgendes befinden sich einige der gängigsten DBMS:
\begin{itemize}
	\item Microsoft SQL Server
	\item MS-Access
	\item MySQL
	\item PostgreSQL
	\item HSQLDB
	\item H2 Derby
	\item Oracle
	\item DB2
	\item Sybase
\end{itemize}
Um ein geeignetes DBMS auszuwählen, gibt es viele Kriterien wie die Ausführungszeit, CPU- und Speicher Nutzung. Der Artikel von Youssif Bassil, A Comparative Study on the Performance of the Top DBMS Systems, im Jahr 2011 vergleicht einige Datenbankmanagementsysteme anhand der genannten Kriterien.
%https://arxiv.org/pdf/1205.2889.pdf
\section{Datenbank Migration}

%- Was allgemeines (1 Satzt)
%- Was ist Datenbank Migration
%- Wieso wird Datenbank Migration benötigt?
%- Welche Arten von Datenbank Migrationen gibt es?



Datenbank Migration wird immer mehr von Unternehmen bzw. Organisationen gebraucht. 

Die Migration von Datenbanken dient zum Verschieben der Daten von der Quell-Datenbank zur Ziel-Datenbank einschließlich die Schemaübersetzung und Datentransformation.


Mögliche Gründe für eine Dantenbank Migration sind:
\begin{itemize}
	\item Upgrade auf eine neue Software oder Hardware
	\item Änderung der Unternehmensrichtlinien
	\item Investition in IT-Diienstleistungen
	\item Integration von Datenquelle in ein System
	\item Zusammenführen mehrerer Datenbanken in einer Datenbank für eine einheitliche Datenansicht.
	\item Wartung des existierenden Systems ist schwer oder nicht möglich.
\end{itemize}

Außerdem gibt es unterschiedliche Strategien für Datenbank Migration. Diese können in drei Kategorien unterteilt werden:

\begin{enumerate}
	\item Migration durch objekt orientierte Schnittstellen: \\
	Bei dieser Strategie werden Daten in form von Objekten bzw. XML Dateien verarbeitet. Dafür wird ein bidirektionales Mapping benötigt, ojektbasierte Schemas in Datenbank Schemas zu übersetzten.
	\item Datenbank Integration: \\
	Hier wird die Quell-Datenbank mit der Ziel-Datenbank verbunden, wodurch der Eindruck entsteht, als ob alle Daten in einer einzigen Datenbank gespeichert sind.	
	\item Datenbank Migration: \\
	Die Quell-Datenbank wird in die Ziel-Datenbank kopiert. Dabei werden Schemas in ein Zielschema semantisch übersetzt werden. Darauf basierend werden die enthaltenen Daten konvertiert.
\end{enumerate}



\section{Verwandte Arbeiten}
Eines der Hauptprobleme in der Softwareindustrie besteht darin, eine hochwertige Datenverwaltung sicherzustellen. Dies ist auch der Fall bei einer Datenbank Migration, wobei die mit dem Migrationsworkflow verbundenen Aufgaben vielfältig und kompliziert sind. Das manuelle Ausführen dieser Aufgaben erfordert viel Zeit und ein sehr erfahrenes Team. Um Zeit und Kosten bei der Migration zu sparen und um wiederholende Aufgaben zu automatisieren, bieten sich zahlreiche Tools bzw. Prototypen für Datenbank Migration (DBMT für Databbase Migration Tool). \\
Einige dieser Tools werden in der Tabelle \ref{table:tools} vorgestellt. Diese basiert sich auf den Vorschlag von Jutta Hortsmann, J.

\begin{table}
\begin{center}
	\begin{tabular}{ |p{3cm}|p{3cm}|p{3cm}|p{2cm}|p{3cm}| }
		\hline
		\textbf{Name} & \textbf{Quell-DBMS} &  \textbf{Ziel-DBMS} &\textbf{Lizenz} & \textbf{Betriebssysteme} \\
		\hline
		 OSDM Toolkit (Apptility) & 
		  Oracle, SyBase, Informix, DB2, MS Access, MS SQL &  PostgreSQL, MySQL & 
		 Frei &  Windows, Linux, Unix und Mac OS \\
		 \hline
		 DB Migration (Akcess) &  Oracle und MS SQL &  PostgreSQL und MYSQL  & Kommerziell & Windows\\
		 \hline
		 Mssql2 Pgsql (OS Project) &   MS SQL&   PostgreSQL  & Frei & Windows \\
		 \hline
		 MySQL
		 Migration
		 Toolkit (MySql AB)&  MS Access und Oracle &  MySQL & Frei & Windows  \\
		\hline
		Open DBcopy (Puzzle ITC) & Alle RDBMS& Alle RDMS & Frei & Betriebssystem- unabhängig \\
		\hline
		Progression DB (Versora) &  MS SQL &  PostgreSQL, MySQL und	Ingres & Frei & Linux und Windows \\
		\hline
		Shift2Ingres (OS Project)& Oracle und DB2 & Ingres & Frei &  Betriebssystem- unabhängig \\
		\hline
		SQLPorter (Real Soft Studio)& Oracle, MS SQL, DB2 und Sybase & MySQL & Kommerziell & Linux, Mac OS und Windows \\
		\hline
		SQLWays (Ispirer) & Alle RDMBS & PostgreSQL und MySQL & Kommerziell & Windows \\
		\hline
		SwisSQL Data Migration Tool (AdventNet)& Oracle, DB2, MS SQL, Sybase und MaxDB & MySQL & Kommerziell & Windows \\
		\hline
		SwisSQL SQLOne Console (AdventNet)& Oracle, MSSQL, DB2, Informix und Sybase & PostgreSQL und MySQL  & Kommerziell & Windows \\
		\hline
		MapForce (Altova) & SQL Server, DB2, MS Access, MySQL und PostgreSQL & SQL Server, DB2, MS Access und Oracle & Kommerziell & Windows, Linux und Mac OS \\
		\hline
		Centerprise Data Integrator (Astera) & SQL Server, DB2, MS Access, MySQL und PostgreSQL& SQL Server, DB2, MS Access, MySQL und PostgreSQL& 
		Kommerziell & Windows\\
		\hline
		DBConvert (DB Convert) & Oracle, DB2, SQLite, MySQL, PostgreSQL, MS Access und Foxpro & Oracle, DB2, SQLite, MySQL, PostgreSQL, MS Access und Foxpro & Kommerziell & Windows \\
		\hline
		SQuirrel DBCopy Plugin (Sourceforge) & Alle RDBMS  & Alle RDBMS & Frei & Alle Betriebssysteme\\
		\hline
	\end{tabular}
\end{center}
\caption{Database Migration Tools}
\label{table:tools}
\end{table}
%http://citeseerx.ist.psu.edu/viewdoc/download?doi=10.1.1.94.3883&rep=rep1&type=pdf (juttaa)


\section{GuttenBase}
Viele Software Unternehmen haben sich dafür entschieden, ein eigenes Tool für Datenbankmigration zu entwickeln. Dies ist der Fall bei der Firma Akquinet AG, wo die Open Source Bibliothek GuttenBase in 2012 entwickelt wurde. Da GuttenBase open source ist, wurde sie in weiteren Schritten weiterentwickelt und um zusätzliche Funktionen erweitert.\\
Anderes als die in der Tabelle \ref{table:tools} vorgestellten Tools, bietet die GuttenBase Bibliothek eine gewisse Flexibilität bei der Migration. Migrationsschritte können durch das Überschreiben der Mapping Klassen spezifiziert werden, damit die Migration passend zu dem aktuellen Stand der Daten ausgeführt wird.\\
Dieser Ansatz erlaubt Entwicklern, eine volle Kontrolle über den Migrationsprozess zu haben.\\
Für die Migration einer Datenbank ist häufig eine benutzerdefinierte Lösung erforderlich. Beispilsweise z. B. das Unbenennen von Tabellen bzw. Spalten in der Zieldatenbank, das Umwandeln von Spaltentypen, das Ausschließen von bestimmten Tabellen bzw. Spalten usw..
In diesem Fall können Konfigurationshinweise vor der Migration hinzugefügt werden. Standardmäßig wird eine Standardimplementierung der Hinweise nach dem Verbinden der Datenbanken hinzugefügt. Diese können jedoch von dem Nutzer überschrieben werden. 
%TODO Tabelle hinzufügen!

%- Allgemeiner Satz
%- Was ist GuttenBase
%- Wann wurde sie eingeführt?
%- Von wem ist sie entwickelt?
%- Warum soll man GuttenBase benutzten
%- Wie kann man GuttenBase benutzen?
%- [Was kann man in GuttenBase optimieren?]



\chapter{Umsetzung}
\section{Analyse}
\subsection{Anforderungen}
- Anforderungen (funktionale + nicht funktionale)\\

\subsection{Probleme und Strategien}
%- Umsetzungsform\\
%- ...
\section{Konzeption}
%- DatenModell: GBActions, DB Elemente\\
%- Abstrakte Mapping \\
%- Multiple Select \\
%- Aktionen speichern \\
%- Protoypen \\
\subsection{Umsetzungsform}
%- allgemeines: warum muss erst die Umsetzungsform entschieden werden (vor der Konzeption). \\
%- welche Alternativen gibt es um ein solches Plugin entwickeln zu können?\\
%- Vorteile und Nachteile jeder Alternative.\\
%- Argumente für IntelliJ Plugin.
\subsubsection{IntelliJ Plugin Entwicklung}
%- Was ist IntelliJ \\
%- was ist IntelliJ Plugin Entwicklung \\
%- wie lässt sich ein Plugin mit IntelliJ entwickeln?\\

\subsection{Konzeptionelle Sicht}
\subsection{Modulsicht}
\subsection{Datensicht}
\subsection{Prototypen}



\section{Implementierung}
\subsection{verwendete Technologien}
- Swing \\
- UI Form \\
- Java\\
- Gutenbase

\subsection{Features}
%- add gbact\\
%- save gb act\\
%- migrate db \\
%- + screenshots usw.



\chapter{Evaluation}
%- Allgemeines: Usability, Kriterien,... 
\section{Expert Interview}
- Interview mit Nicole
\section{user test}
- Umfrage 
\chapter{Fazit und Ausblick}
%Das zu Beginn der Bachelorarbeit gesetzte Ziel wurde mithilfe des GuttenBase Plugin erreicht. \\
%Mit dem GuttenBase Plugin lassen sich Datenbanken durch wenige Klicke migrieren. Während des Migrationsprozesses können Konfigurationsschritte hibzugefügt werden wie das Umbenennen der Quell-Tabellen bzw. Spalten sowie das Ausschließen bestimmter Tabellen bzw. Spalten und das Ändern von Spalten-Datentypen. Diese können vor- oder in dem  Migrationsprozess erstellt werden und werden gespeichert, um sie bei der nächsten Migration wiederverwenden zu können.\\
%Das GuttenBase Plugin läuft basierend auf der GuttenBase Bibliothek und interagiert mit dem Database Plugin von IntelliJ.\\
%\\ \\

\section{Fazit}
Nach einer Einführung in die Grundlagen der Datenbank Migration, Guttenbase und die IntelliJ Plugin Entwicklung, wurde eine Anforderungsanalyse derchgeführt. basierend darauf wurde das GuttenBase Plugin entworfen, implementiert und evaluiert. \\
Mit dem GuttenBase Plugin lassen sich Datenbanken durch wenige Klicke migrieren. Während des Migrationsprozesses können Migrationsoperationen hibzugefügt werden wie das Umbenennen der Quell-Tabellen bzw. Spalten sowie das Ausschließen bestimmter Tabellen bzw. Spalten und das Ändern von Spalten-Datentypen. Diese können vor- oder in dem  Migrationsprozess erstellt werden und werden gespeichert, um sie bei der nächsten Migration wiederverwenden zu können.\\
Das GuttenBase Plugin läuft basierend auf der GuttenBase Bibliothek und interagiert mit dem Database Plugin von IntelliJ.\\
Mit den oben genannten Funktionalitäten erreicht das GuttenBase Plugin das zu Beginn der Bachelorarbeit gesetztes Ziel.
\section{Ausblick}

%\begin{thebibliography}{9}
%	\bibitem{latexcompanion} 
%	Michel Goossens, Frank Mittelbach, and Alexander Samarin. 
%	\textit{The \LaTeX\ Companion}. 
%	Addison-Wesley, Reading, Massachusetts, 1993.
%	\bibliography{bibliography}
%\end{thebibliography}

\printbibliography


\printbibliography[heading=bibintoc,title={Literatur}]
%\printglossary[style=thesisacr,type=acronym,title=\abbreviationsname]
%\printglossary[style=thesis,type=main,title=\glossaryname]
\end{document}
