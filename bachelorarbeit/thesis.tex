%!TEX TS-program = xelatex
%!TEX encoding = UTF-8 Unicode
\documentclass[
%12pt, % Schriftgröße
%DIV10, % Änderung der Größe des Satzspiegels (bedruckbarer Bereich einer Seite), nur in Verbindung mit koma-script verwendbar
ngerman, % für Umlaute, Silbentrennung etc.
%a4paper, % Papierformat
%oneside, % einseitiges Dokument
%titlepage, % es wird eine Titelseite verwendet
%parskip=half, % Abstand zwischen Absätzen (halbe Zeile)
%headings=normal, % Größe der Überschriften verkleinern
%listof=totoc, % Verzeichnisse im Inhaltsverzeichnis aufführen
%bibliography=totoc, % Literaturverzeichnis im Inhaltsverzeichnis aufführen
%index=totoc, % Index im Inhaltsverzeichnis aufführen
%captions=tableheading, % Beschriftung von Tabellen unterhalb ausgeben
%final
]{thesis}

\usepackage[chapter]{tocbibind}

\usepackage{blindtext}
\usepackage{float}

\usepackage{glossaries}
\usepackage{biblatex}

%\setStyleFile{glossary}
\makeglossaries
 \newglossaryentry{label}{
   name={name1},
   description={long description}
 }
 \newglossaryentry{TEST}{
	name={name2},
	description={long description}
}
 \newglossaryentry{ALOO}{
	name={name3},
	description={long description}
}


\input{acronyms.tex}
\bibliography{bib.bib}
\makeatletter

%%% USER SETTINGS

\author{Sirajeddine Ben Zinab}
\matriculationnumber{3094966}

\title{Entwicklung Einer Anwendungsoberfläche Für Datenbankmigration Mit GuttenBase}
\subtitle{}

\date{28.\,Februar 2021}

\institute{Universität Bremen}
\department{Fachbereich 3: Mathematik und Informatik}

\examiner{Prof. Dr. Sebastian Maneth}
\supervisor{Prof. Dr. Martin Gogolla}


%%% DOCUMENT SETTINGS

% uncomment for 1.5-linespacing
\onehalfspacing

\makeatother
\endinput
%% anything below this line is ignored

\setcounter{secnumdepth}{4}

\begin{document}
% Titel
%\maketitle
% Impressumsseite
\clearpage
\thispagestyle{empty}\small
\null\vfill

\makeatletter
  \textbf{\@author}\par
  \@title\par
  \ifthesis@hassubtitle\@subtitleshort\fi

  %\dfrac{diploma}{den}name, \@department\par
  \@institute, \monthyear
\makeatother

\normalsize

\clearpage
\input{chapters/declarations.tex}
%
%\clearpage
%\section*{Danksagung}
An dieser Stelle möchte ich allen Menschen meinen Dank aussprechen, die mir geholfen und mich dabei unterstützt haben, diese Arbeit zu erstellen. \\
Als erstes möchte ich mich bei meinem Betreuer Prof. Dr. Sebastian Maneth bedanken, der mich während meiner Bachelorarbeit unterstützt und mir wertvolle Tipps gegeben hat.\\
Ich bedanke mich außerdem bei Herrn Kolja Koischwitz und Herrn Markus Dahm von der Firma akquinet AG für die Kooperation. Die freundliche und zuverlässige Zusammenarbeit hat mir eine praxisnahe Forschung ermöglicht und wertvolle Einblicke gewährt.\\
Letztlich richte ich auch ein Dankeschön an Herr Ahmed Azzouz für das Korrekturlesen meiner Arbeit sowie an meine Eltern und Freunde, die mir während meines ganzen Studium unterstützt haben.

\clearpage
\section*{Zusammenfassung}

Migration ist in der Wissenschaft kein neues Thema. Es existieren zahlreiche Methoden und Frameworks zur Beschreibung, Analyse und Implementierung der Migration. Dies gilt auch für Datenbankverwaltungssysteme (DBMS). \\ 
In dieser Arbeit werden aktuelle Tools für die Datenbankmigration vorgestellt. Dabei werden bedeutsamen Eigenschaften der Open-Source-Bibliothek GuttenBase erläutert.\\
Außerdem hat diese Arbeit hauptsächlich den Entwurf, die Implementierung und die Evaluation eines Tools für Datenbank Migration zwischen verschiedenen Datenbanksystemen (DBMS) basierend auf GuttenBase zum Thema.
Um die Nutzung der GuttenBase-Bibliothek für möglichst viele Nutzer zu ermöglichen, erfolgt die Umsetzung als ein IntelliJ Plugin (IDEA).\\ 
Diese Bachelorarbeit wurde bei der Firma Akquinet AG in Bremen im Zeitraum von September 2020 bis zum März 2021 erstellt und stellt den Abschluss meines Bachelorstudiums an der Universität Bremen dar. \\ 
Der Text liegt in deutscher Sprache vor.







\frontmatter
\phantomsection
%\addcontentsline{toc}{section}{\contentsname}
\tableofcontents




\cleardoublepage
\listoffigures

\cleardoublepage
\listoftables

%\clearpage
%\listoftodos

\cleardoublepage
\mainmatter

\chapter{Einleitung}


\section{Problemstellung}

Problemstellung/Forschungsfrage: 
Wie lässt sich ein Tool für DB Migration basierend auf GB entwickeln?
Was sind die aktuellen Tools für Datenbank Migration?
Wie lässt sich die GB Bibliothek optimieren?


\section{Motivation}
Guttenbase entwicklung 2012 \\
Guttenbase Anpassung bis 2018\\
Guttenbase hat viele Vorteile\\
Aber seit 2018 keine Optimierung\\
Auch wenn Guttenbase für Entwickler entwickelt wurde, ist die Benutzung von Guttenbase nicht optimal.\\
In Guttenbase muss man bei jeder Migration das ganze Mapping selber implementieren (überschreiben). \\
Außerdem fehlt die Möglichkeit, eine Übersicht der Datenbank zu haben und Konfigurationsschritte hinzuzufügen, während man migriert.
Aus diesen Gründen lässt sich die Nutzung der GuttenBase Bibliothek einschränken. Deswegen bietet sich die Möglichkeit, ein Tool zu realisieren, um die Nutzung der GuttenBase Bibliothek zu optimieren und eine flexible, einfache und konfigurierbare Datenbank Migration zu ermöglichen.

\section{Zielsetzung}
Guttenbase Tool umsetzen und Evaluieren und aktuelle Tools für Datenbank Migration vergleichen.
\section{Aufbau der Arbeit}
Zu Beginn der Arbeit werden ein Paar Begriffe erlautert.\\
Außerdem wird die GuttenBase Bibliothek kurz vorgestellt.\\
Es werden zusätzlich aktuelle Tools für Datenbank Migration vorgestellt und miteinander verglichen.\\
Der Hauptteil dieser Arbeit beschäftigt sich mit der Umsetzung des Guttenbase Tool. Dabei wird zuerst die Umsetzungsform begründet und es werden die Schritte der Umsetzung denauer dargestellt.\\
Danach wird das Ergebnis kurz evaluiert und am Ende gibt es offene Fragen sowie Optimierungsmöglichkeiten geben.
\chapter{Grundlagen}
Dieses Kapitel liefert einen allgemeinen Einblick einige Grundaspekte der Datenbank Migration sowie der GuttenBase Bibliothek. Außerdem werden verwandte Arbeiten vorgestellt. 
\section{Datenbanken}
%- allgemeine Def
%- Arten von Datenabnken
%https://books.google.de/books?hl=de&lr=&id=_3XVAwAAQBAJ&oi=fnd&pg=PA14&dq=Datenbank+Grundlagen+&ots=ntu4qhQNQD&sig=LpAgd_s5f04ulLTfOYe6lEt1_Zc#v=onepage&q=Datenbank%20Grundlagen&f=false
Datenbanken spielen seit der Neuerung des IT-Zeitalter eine wichtige Rolle in dem elektronischen Datenmanagement.\\
Eine Datenbank ist eine geordnete, selstbeschreibende Sammlung von Daten, die miteinander in Beziehung stehen.
Vielmehr ist eine Datenbank ein verteiltes, integriertes Computersystem, das Nutzdaten und Metadaten enthält. Nutzdaten sind dabei die Daten, die Benutzer in der Datenbank anlegen und aus denen die Informationen gewonnen werden. Metadaten werden of auch als Daten über Daten bezeichnet und helfen, die Nutzdaten der Datenbank zu strukturieren. 
	
	
	
%\section{Datenbank Management System (DBMS)}
%- Def
%- Beispiele
Damit Datenbanken auf einem Computer verwaltet werden können, werden Datenbankmanagement Systeme (DBMS) benötigt. Diese sind leistungsfähige Programme für die flexible Speicherung und Abfrage strukturierter Daten. \\
Außerdem hilft ein DBMS bei der Organisation und Integrität von Daten und regelt den Zugruff auf Datengruppen. \\
Ein DBMS kann aus einem einzelenen Programm bestehen. Dies ist z. B. bei einem Desktop-DBMS zu sehen. Es kann jedoch aus verschiedenen Programmen bestehen, die zusammenarbeiten und die Funktion des DBMS bereitstellen. Dies ist z. B. bei den servergestützen Datenbanksystemen der Fall.\\
Um eine Datenbank Anwendung zu implementieren, sollte auf das Datenbankmodell geachtet werden. Dies stellt die Daten einer Datenbank und deren Beziehungen abstrakt dar. Meistens wird ein relationales Datenbankmodell eingesetzt. Dies hat, im Gegensatz zu den anderen Datenbankmodellen, keine strukturelle Abhängigkeit und versteckt die physikalische Komplexität der Datenbank komplett vor den Anwendern.\\
Es stehen zahlreiche Datenbankmanagementsysteme zur Verfügung. Folgendes befinden sich einige der gängigsten DBMS:
\begin{itemize}
	\item Microsoft SQL Server
	\item MS-Access
	\item MySQL
	\item PostgreSQL
	\item HSQLDB
	\item H2 Derby
	\item Oracle
	\item DB2
	\item Sybase
\end{itemize}
Um ein geeignetes DBMS auszuwählen, gibt es viele Kriterien wie die Ausführungszeit, CPU- und Speicher Nutzung. Der Artikel von Youssif Bassil, A Comparative Study on the Performance of the Top DBMS Systems, im Jahr 2011 vergleicht einige Datenbankmanagementsysteme anhand der genannten Kriterien.
%https://arxiv.org/pdf/1205.2889.pdf
\section{Datenbank Migration}

%- Was allgemeines (1 Satzt)
%- Was ist Datenbank Migration
%- Wieso wird Datenbank Migration benötigt?
%- Welche Arten von Datenbank Migrationen gibt es?



Datenbank Migration wird immer mehr von Unternehmen bzw. Organisationen gebraucht. 

Die Migration von Datenbanken dient zum Verschieben der Daten von der Quell-Datenbank zur Ziel-Datenbank einschließlich die Schemaübersetzung und Datentransformation.


Mögliche Gründe für eine Dantenbank Migration sind:
\begin{itemize}
	\item Upgrade auf eine neue Software oder Hardware
	\item Änderung der Unternehmensrichtlinien
	\item Investition in IT-Diienstleistungen
	\item Integration von Datenquelle in ein System
	\item Zusammenführen mehrerer Datenbanken in einer Datenbank für eine einheitliche Datenansicht.
	\item Wartung des existierenden Systems ist schwer oder nicht möglich.
\end{itemize}

Außerdem gibt es unterschiedliche Strategien für Datenbank Migration. Diese können in drei Kategorien unterteilt werden:

\begin{enumerate}
	\item Migration durch objekt orientierte Schnittstellen: \\
	Bei dieser Strategie werden Daten in form von Objekten bzw. XML Dateien verarbeitet. Dafür wird ein bidirektionales Mapping benötigt, ojektbasierte Schemas in Datenbank Schemas zu übersetzten.
	\item Datenbank Integration: \\
	Hier wird die Quell-Datenbank mit der Ziel-Datenbank verbunden, wodurch der Eindruck entsteht, als ob alle Daten in einer einzigen Datenbank gespeichert sind.	
	\item Datenbank Migration: \\
	Die Quell-Datenbank wird in die Ziel-Datenbank kopiert. Dabei werden Schemas in ein Zielschema semantisch übersetzt werden. Darauf basierend werden die enthaltenen Daten konvertiert.
\end{enumerate}

\section{IntelliJ Plugin Entwicklung}
%- IntelliJ, IntelliJ IDEA, allgemeine Infos
%- Plugin Entwicklung
%- Erweiterungspunkte
%- Action System
%- DB Plugin
Die von der Firma JetBrains und in Java entwickelte Entwicklungsumgebung (IDE), IntelliJ IDEA, ist ein Teil einer Reihe von ähnlich JetBains Entwicklungsumgebungen wie Clion, PyCharm, PhpStrom, DataGrip usw. Diese basieren sich auf den gleichen Kern, nämlich den IntelliJ SDK.\\
IntelliJ IDEA ist mit der Implementierung in Java, Kotlin, Groovy und Scala geeignet. Sie enthält unterschiedliche Funktionen



\section{Verwandte Arbeiten}
Eines der Hauptprobleme in der Softwareindustrie besteht darin, eine hochwertige Datenverwaltung sicherzustellen. Dies ist auch der Fall bei einer Datenbank Migration, wobei die mit dem Migrationsworkflow verbundenen Aufgaben vielfältig und kompliziert sind. Das manuelle Ausführen dieser Aufgaben erfordert viel Zeit und ein sehr erfahrenes Team. Um Zeit und Kosten bei der Migration zu sparen und um wiederholende Aufgaben zu automatisieren, bieten sich zahlreiche Tools bzw. Prototypen für Datenbank Migration (DBMT für Databbase Migration Tool). \\
Einige dieser Tools werden in der Tabelle \ref{table:tools} vorgestellt. Diese basiert sich auf den Vorschlag von Jutta Hortsmann, J.

\begin{table}
\begin{center}
	\begin{tabular}{ |p{3cm}|p{3cm}|p{3cm}|p{2cm}|p{3cm}| }
		\hline
		\textbf{Name} & \textbf{Quell-DBMS} &  \textbf{Ziel-DBMS} &\textbf{Lizenz} & \textbf{Betriebssysteme} \\
		\hline
		 OSDM Toolkit (Apptility) & 
		  Oracle, SyBase, Informix, DB2, MS Access, MS SQL &  PostgreSQL, MySQL & 
		 Frei &  Windows, Linux, Unix und Mac OS \\
		 \hline
		 DB Migration (Akcess) &  Oracle und MS SQL &  PostgreSQL und MYSQL  & Kommerziell & Windows\\
		 \hline
		 Mssql2 Pgsql (OS Project) &   MS SQL&   PostgreSQL  & Frei & Windows \\
		 \hline
		 MySQL
		 Migration
		 Toolkit (MySql AB)&  MS Access und Oracle &  MySQL & Frei & Windows  \\
		\hline
		Open DBcopy (Puzzle ITC) & Alle RDBMS& Alle RDMS & Frei & Betriebssystem- unabhängig \\
		\hline
		Progression DB (Versora) &  MS SQL &  PostgreSQL, MySQL und	Ingres & Frei & Linux und Windows \\
		\hline
		Shift2Ingres (OS Project)& Oracle und DB2 & Ingres & Frei &  Betriebssystem- unabhängig \\
		\hline
		SQLPorter (Real Soft Studio)& Oracle, MS SQL, DB2 und Sybase & MySQL & Kommerziell & Linux, Mac OS und Windows \\
		\hline
		SQLWays (Ispirer) & Alle RDMBS & PostgreSQL und MySQL & Kommerziell & Windows \\
		\hline
		SwisSQL Data Migration Tool (AdventNet)& Oracle, DB2, MS SQL, Sybase und MaxDB & MySQL & Kommerziell & Windows \\
		\hline
		SwisSQL SQLOne Console (AdventNet)& Oracle, MSSQL, DB2, Informix und Sybase & PostgreSQL und MySQL  & Kommerziell & Windows \\
		\hline
		MapForce (Altova) & SQL Server, DB2, MS Access, MySQL und PostgreSQL & SQL Server, DB2, MS Access und Oracle & Kommerziell & Windows, Linux und Mac OS \\
		\hline
		Centerprise Data Integrator (Astera) & SQL Server, DB2, MS Access, MySQL und PostgreSQL& SQL Server, DB2, MS Access, MySQL und PostgreSQL& 
		Kommerziell & Windows\\
		\hline
		DBConvert (DB Convert) & Oracle, DB2, SQLite, MySQL, PostgreSQL, MS Access und Foxpro & Oracle, DB2, SQLite, MySQL, PostgreSQL, MS Access und Foxpro & Kommerziell & Windows \\
		\hline
		SQuirrel DBCopy Plugin (Sourceforge) & Alle RDBMS  & Alle RDBMS & Frei & Alle Betriebssysteme\\
		\hline
	\end{tabular}
\end{center}
\caption{Database Migration Tools}
\label{table:tools}
\end{table}
%http://citeseerx.ist.psu.edu/viewdoc/download?doi=10.1.1.94.3883&rep=rep1&type=pdf (juttaa)


\section{GuttenBase}
\label{section:grundlagen:gb}
Viele Software Unternehmen haben sich dafür entschieden, ein eigenes Tool für Datenbankmigration zu entwickeln. Dies ist der Fall bei der Firma Akquinet AG, wo die Open Source Bibliothek GuttenBase in 2012 entwickelt wurde. Da GuttenBase open source ist, wurde sie in weiteren Schritten weiterentwickelt und um zusätzliche Funktionen erweitert.\\
Anderes als die in der Tabelle \ref{table:tools} vorgestellten Tools, bietet die GuttenBase Bibliothek eine gewisse Flexibilität bei der Migration. Migrationsschritte können durch das Überschreiben der Mapping Klassen spezifiziert werden, damit die Migration passend zu dem aktuellen Stand der Daten ausgeführt wird.\\
Dieser Ansatz erlaubt Entwicklern, eine volle Kontrolle über den Migrationsprozess zu haben.\\
Für die Migration einer Datenbank ist häufig eine benutzerdefinierte Lösung erforderlich. Beispilsweise z. B. das Unbenennen von Tabellen bzw. Spalten in der Zieldatenbank, das Umwandeln von Spaltentypen, das Ausschließen von bestimmten Tabellen bzw. Spalten usw..
In diesem Fall können Konfigurationshinweise vor der Migration hinzugefügt werden. Standardmäßig wird eine Standardimplementierung der Hinweise nach dem Verbinden der Datenbanken hinzugefügt. Diese können jedoch von dem Nutzer überschrieben werden. 
%TODO Tabelle hinzufügen!

%- Allgemeiner Satz
%- Was ist GuttenBase
%- Wann wurde sie eingeführt?
%- Von wem ist sie entwickelt?
%- Warum soll man GuttenBase benutzten
%- Wie kann man GuttenBase benutzen?
%- [Was kann man in GuttenBase optimieren?]



\chapter{Umsetzung}
Dieses Kapitel erläutert alle Schritte der Umsetzung des GuttenBase Plugins beginnend mit der Analyse bis zu zur Architektur und Implementierung.
\section{Analyse}
Am Anfang dieses Abschnitts wird die Umsetzungsform des GuttenBase Plugins festgelegt, da diese einen Einfluss auf die Architektur sowie die zu verwendenen Technologien hat. Außerdem werden die Anforderungen an das System definiert.\\
Um den Soll-Zustand genauer zu definieren werden Prototypen eingesetzt.
\subsection{Umsetzungsform}
Um eine optimale Nutzung des GuttenBase Plugins zu erzielen, soll auf die Umsetzungsform geachtet werden.\\
Das zu entwickelnde Tool kann z. B. als eine Desktop Applikation, Web Applikation oder als Plugin einer anderen Anwendung realisiert werden.\\
In der Tabelle \ref{table:tool-options} werden einige Vor- und Nachteie jeder Alternative erläutert. \\
Alle drei Alternativen haben Pros und Contras allerdings ist die schnellere Erreichung von vielen Nutzern sowie die Einfache Installation bei der IDE Plugin Entwicklung entscheidend. 
\begin{table}
	\centering
		\begin{tabular}{ |p{3cm}|p{6cm}|p{6cm}| }
			\hline
			\textbf{Alternative} & \textbf{Vorteile} &  \textbf{Nachteile}  \\
			Desktop App & 
			\begin{itemize}
				\item Offline immer verfügbar
				\item Volle Kontrolle über die Anwendung und die enthaltenen Daten.
				\item Bessere Leistung, da kein Browser als Zwischenschicht existiert.
			\end{itemize}& 
			\begin{itemize}
				\item Platformabhängig
				\item Hohe Entwicklungskosten
				\item Installation ist notwendig
			\end{itemize} \\
			\hline
			Web App &
			
			\begin{itemize}
				\item Installation oder manuelle Updates sind nicht notwändig. 
				\item geringere Entwicklung- und Wartungskosen, da die Anwendung unabhängig von lokalen Endgeräten ist.
			\end{itemize} &
		
			\begin{itemize}
				\item Offline meistens nicht verfügbar.
				\item Geringere Leistung.
				\item Es kann auf bestimmte Gerätehardware nicht zugegriffen werden.
			\end{itemize} \\
			\hline
			IDE Plugin Entwicklung &
			
			\begin{itemize}
				\item Viele Nutzer können erreicht werden.
				\item Einfach zu installieren.
				\item Manche Komponenten bzw. Funktionalitäten der zu erweiternden IDE können wiederverwwendet werden, was die Entwicklungsdauer verkürzt.
				\item Intuitive Nutzung sowie eine einheitliche Benutzeroberfläche wie die benutzte IDE.
			\end{itemize} &
			
			\begin{itemize}
				\item Einarbeitung in die Plugin Entwicklung der ausgewählten IDE ist erforderlich.
				\item Die Flexibilität beim Entwickeln ist durch die limitierte Erweiterbarkeit der IDE eingeschränkt.
			\end{itemize} \\
			\hline
		
		\end{tabular}
	\caption{Umsetzungsmöglichkeiten}
	\label{table:tool-options}
\end{table}

Zunächst soll für eine konkrete IDE entschieden werden. Um diese auszuwählen, muss auf die Anzahl der Nutzer, die Verfügbarkeit der Dokumentation für Plugin Entwicklung sowie die Unterstützung von Datenbanken geachtet werden.\\
Einer der bekanntesten Methoden, um die Beliebtheit einer Programmiersprache bzw. eine IDE herauszufinden, ist der PYPL-Index. Er basiert sich auf Rohdaten aus Google Trends. PYPL enthält den TOP-IDE-Index, welches alanysiert, wie oft IDEs bei Google durchgesucht werden. Die Suchanfragen spiegeln zwar nicht unbedingt die Beliebtheit der IDEs spiegeln. Allerdings hilft einen solchen Index bei der Wahl einer Entwicklungsumgebung.
Bei dieser Analyse sind die drei bekanntesten und für unseren Fall relevanten Entwicklungsumgebungen Visual Studio (erster Platz), Ecllipse (zweiter Platz) und IntelliJ (sechster Platz). Außerdem hat sich der Index von IntelliJ IDE am stärksten erhöht (siehe Abbildung \ref{img:ide-index})\\
Bei eine anderen Umfrage (Jaxenter), mit welcher Entwicklungsumgebung am liebsten in Java programmiert wird, war IntelliJ sogar im ersten Platz mit 1660 Stimmen von 2934. \\

\begin{figure}[h]
	\caption{Top IDE Index}
	\centering
	\includegraphics[width=0.8\textwidth]{images/ide-index}
	\label{img:ide-index}
\end{figure}
Aus den oben fgenannten Gründen wird das GuttenBase als ein Intellij Plugin umgesetzt werden. 

\subsection{Allgemeine Beschreibung der Anforderungen}
Die Anforderungsanalyse ergab folgende Punkte, die von dem GuttenBase Plugin erfüllt werden sollen:
\begin{enumerate}
	\item \textbf{Konfigurationsschritte verwalten:}\\
	Um den Migrationsprozess zu individualisieren, soll der Benutzer die Möglichkeit haben, neue Konfigurationsschritte zu erstellen, zu editieren und zu löschen.
	\item \textbf{Konfigurationsschritte speichern:} \\
	hinugefügte Konfigurationsschritte sollen nach Bestätigung vom Benutzer gespeichert werden können. Diese sollen auch nach einem Neustart der Anwendung zur Verfügen stehen.
	\item \textbf{Überblick über alle Konsfigurationsschritte:}
	Der Benutzer soll über eine tabellarische Auflistung aller erstellten Konfigurationsschritte haben.
	\item \textbf{Datenbanken Verbiden:} \\
	Um eine erfolgreiche Migration durchzuführen, soll der Benutzer in der Lage sein, eine Verbindung zwischen der Quell- und Ziel-Datenbank herzustellen. Die zu migrierende Datenbank sowie die Ziel-Datenbank sollen aus den existierenden Datenbanken ausgewählt werden können.
	\item \textbf{Überblick über enthaltene Datenbankelemente:}\\
	Während des Pigrationsprozess, soll der Benutzer einen Überblick über alle in der Quell-Datenbank enthaltenen Tabellen bzw. Spalten verfügen.
	\item \textbf{Existierende Konfigurationsschritte zur Migration hinzufügen:}\\
	Gespeicherte Konfigurationsschritte sollen bei der Übersicht der Datenbank Elementen zur Verfügung stehen. Diese können auf die entsprechenden Datenbank Elementen angewendet werden.
	\item \textbf{Hinzugefügte Konfigurationsschritte löschen:}\\	
	Der Benutzer soll die Möglichkeit haben, hinzugefügte Konfigurationsschritte zu löschen, nachdem sie zur Migration hinzugefügt wurden.
	\item \textbf{Migrationssprozess starten:} \\
	Im letzten Schritt der Migration kann der Benutzer den Migrationsprozess mit den hinzugefügten Konfigurationsschritten starten.	
	\item \textbf{Überblick über den Fortschritt der Migrationsprozess:}\\
	Damit der Benutzer den Migrationsprozess verfolgen kann, soll einen Überblick über den Fortschritt zur verfügung stehen. 
\end{enumerate}
Konfigurationsschritte beziehen sich hauptsächlich auf die Hinweise der GuttenBase Bibliothek. Um den Umfang dieser Arbeit in Grenzen zu halten, wurden folgende wichtige Konfigurationsschritte für die Umsetzung ausgewählt:

\begin{enumerate}
	\item Spalten  umbenennen.
	\item Tabellen umbenennen.
	\item Filteroptionen für Spalten hinzufügen.
	\item Filteroptionen für Tabellen hinzufügen.
	\item Datentypen von Spalten ändern.
\end{enumerate}

\subsection{detaillierte Beschreibung der Anforderungen}
Dieser Abschnitt beschäftigt sich mit den zu implementierenden Anforderungen. Diese decken den wichtigsten Funktionsumfang des Systems.\\
Neben der textuellen Beschreibung werden auch Anwendungsfalldiagramme erstellt. Dabei wird nur ein Akteur identifiziert. Dieser ist der Benutzer, der die Datenbank Migration durchführt.
\subsubsection{Konfigurationsschritt \textbf{Unbenennen} hinzufügen}
Dieser Anwendungsfall bildet den Vorgang ab, wenn ein Benutzer einen neuen Konfigurationsschritt für das Umbenennen von Spalten bzw. Tabellen in der Ziel-Datenbank. Dies passiert nachdem der Benutzer die Option Umbenennen auswählt und dann
\subsubsection{Konfigurationsschritt \textbf{Excludieren} hinzufügen}
\subsubsection{Konfigurationsschritt \textbf{Spaltentypen Ändern} hinzufügen}
\subsubsection{Konfigurationsschritte verwalten}
\subsubsection{Datenbank Migration durchführen}

\subsection{Prototypen}
%\subsection{Probleme und Strategien}
%- Umsetzungsform\\
%- ...
\section{Konzeption}
%- DatenModell: GBActions, DB Elemente\\
%- Abstrakte Mapping \\
%- Multiple Select \\
%- Aktionen speichern \\
%- Protoypen \\
%- allgemeines: warum muss erst die Umsetzungsform entschieden werden (vor der Konzeption). \\
%- welche Alternativen gibt es um ein solches Plugin entwickeln zu können?\\
%- Vorteile und Nachteile jeder Alternative.\\
%- Argumente für IntelliJ Plugin.
%- Was ist IntelliJ \\
%- was ist IntelliJ Plugin Entwicklung \\
%- wie lässt sich ein Plugin mit IntelliJ entwickeln?\\

\subsection{Anwendungsfälle}
\subsection{Konzeptionelle Sicht}
\subsection{Modulsicht}
\subsection{Datensicht}



\section{Implementierung}
\subsection{verwendete Technologien}
- Swing \\
- UI Form \\
- Java\\
- Gutenbase
\subsubsection{IntelliJ Plugin Entwicklung}

\subsection{Features}
%- add gbact\\
%- save gb act\\
%- migrate db \\
%- + screenshots usw.



\chapter{Evaluation}
%- Allgemeines: Usability, Kriterien,... 
\label{sec:evaluation}



Die folgenden Abschnitte beschäftigen sich mit der Bewertung des erstellten Plugins. Zuerst wird die Evaluationsmethodik sowie der Ablauf der Evaluation beschrieben und anschließend werden die Ergebnisse vorgestellt.
\section{Evaluationsmethodik}
Um einen Gesamteindruck über die Gebrauchstauglichkeit (Usability) zu haben und Anpassungsmöglichkeiten zu identifizieren, wurde das entwickelte Tool in Form eines qualitativen Experteninterviews evaluiert. 

Es werden in der Regel mindestens fünf Testpersonen benötigt um die meisten auftretenden Probleme (85\%) in einem System zu identifizieren\cite{nielsen1993mathematical}. Allerdings wurde im rahmen dieser Bachelorarbeit  aufgrund beschränkter zeitlicher Ressourcen nur ein Experteninterview durchgeführt. 

die wichtigste Voraussetzung bei der Auswahl der Expertin war lediglig die Erfahrung mit Benutzerschnittstellen und User Experience. Außerdem waren Kenntnisse in der Datenbankverwaltung bzw. Datenbankmigration von Vorteil. 

\section{Durchführung}
Üblicherweise wird das Experteninterviews Leitfadeninterview durchgeführt \cite{mayring1994qualitative}. Dies ermöglicht die Erhebung von einzelnen, bestimmbaren Informationen sowie die Verfolgung von bestimmten Informationszielen. Deswegen wurde diese Befragungsart für die Evaluation gewählt. Bei dieser Befragungsmethode werden oft offene Fragen gestellt, um mehr Freiraum für Antworten und möglichst viele Informationen aus dem Interview zu gewinnen.

Das Experteninterview hat online über Teams\footnote{https://teams.microsoft.com/} stattgefunden. Der Hauptgrund dafür war die aktuelle Corona Pandemie. Außerdem bieten Online Interviews eine flexible Vereinbarung gegenüber den physischen Interviews.

Während des Experteninterviews wurde das GuttenBase Plugin über ein Share-Screen gezeigt. Dabei wurden die wichtigsten Funktionalitäten vorgestellt. 
Zunächst wurden Fragen aus dem Leitfaden gestellt, welches vorab erstellt wurde. Das Leitfaden beinhaltet hauptsächlich Fragen über die Grundsätze der Informationsdarstellung und Benutzer-System-Interaktion, welche im Abschnitt \ref{sec:ziel} erläutert wurden. Außerdem wurden Fragen hinsichtlich der Verbesserungsmöglichkeiten gestellt.




%
%Struktur:
%- was : expert interview
%- Warum: Gesamteindruck + Verbesserungsvorschläge -> schnelle Anpassungen
%- nur 1 expert intervieww (keine zeit)
%
%
%
%
%
%\cite{franziska18interview}
%- : experteninterview (nur 1, da keine Zeit), auchh wenn 5 mindestens benötigt werden \cite{nielsen1993mathematical}
%- Ziel: Machbarkeit + schnelle Anpassungen 
%%- Wie: Proof of Concept??
%- Befragungsmethode: Leitfadeninterview (ziel: Informationsziel verfolgen) - Semistrukturiertes Interview - offene Fragen
%- Auswahl Expertin: Erfahrung in UX/UI + Kenntnisse in DBMS.
%- Inhalt:
%	- Ersteindruck
%	- Usability Grundsätze
%	- Vrbesserungsvorschläge
%- Ablauf: Online Meetin + Screensharing + Zugriff auf Screen
%- Interview Durchführung

\section{Ergebnisse}
Um das Interview auswerten zu können, wurden die Gespräche nach der Durchführung transkribiert. Zunächst wurde das transkribierte Interview  gemäß der qualitativen Inhaltsanalyse kodiert \cite{mayring1994qualitative}. Dabei wurde das Transkript genau durchgelesen und den Textfragmenten wurden bestimmte Kodes zugeordnet.
Nach der Transkription erfolgte die Kategorienbildung. Dabei wurden folgende Kategorien identifiziert, die relevante Kodes zusammenfassen:
\begin{itemize}
	\item Eindruck
	\item Usability
	\item Verbesserungsvorschläge
\end{itemize}

Im Folgenden werden die Ergebnisse des Interviews entsprechend der genannten Kategorien zusammen gefasst.

\subsection*{\textbf{Eindruck}}
Der Gesamteindruck der Anwendungsoberfläche bezeichnet die Expertin als zufriedenstellend und positiv. Die Benutzeroberfläche wirkte insgesamt intuitiv und einheitlich mit der IntelliJ Entwicklungsumgebung. Außerdem findet die Expertin die Datenbankmigration in mehreren Schritten relativ einfach. 
\subsection*{\textbf{Usability}}
Es wurden mehrere Aspekte der Gebrauchstauglichkeit von den Teilnehmerin bewertet. Beispielsweise bewertet sie die \textbf{Fehlertoleranz} positiv. Sie gibt an, dass an fast allen Stellen wird eine Fehlermeldung angezeigt, die die Ursache des Fehlers erklärt. Dies ist der Fall bei der Datenbankverbindung zu sehen, wenn die Zugangsdaten nicht stimmen. Dabei wird die Fehlermeldung allerdings relativ spät angezeigt (erst nach dem Klick auf das Next Button), was die Expertin als verbesserungswürdig bezeichnet.

Die \textbf{Selbstbeschreibfähigkeit} bezeichnet die Teilnehmerin als verbesserungsfähig. Dies liegt haupsächlich an der Platzierung des Plus Button bei der Konfigurationsübersicht des Plugins. Die Teilnehmerin findet konnte nicht herausfinden, was die Funktion des Buttons ist. Außerdem gibt sie an, dass die Bezeichnungen von den Menüpunkten für das Öffnen der Benutzeröberfläche nicht selbstbeschreibend sind.

Die Teilnehmerin bewertet die \textbf{Aufgabenangemessenheit} generell positiv. Allerdings findet sie das Hinzufügen von Migrationsoperationen während der Migration wichtig. Dies wird aktuell allerdings nicht unterstützt.

Die \textbf{Konsistenz} äußert sich die Expertin insgesamt positiv. Sie findet das Trennen von den Quell- und Zieldatenbanken bei der Datenbankverbindung gut. Allerdings ist ihr aufgefallen, dass die Schriftgröße von den \glqq Source\grqq\, und \glqq Target\grqq\, Überschriften zu klein. Sie findet außerdem Die Gruppierung von den Menüpunkten beim Öffnen der Benutzeröberfläche verbesserungswürdig. 

Die Teilnehmerin bezeichnet die \textbf{Erwartungskonformität} als relativ gut. Sie gibt an, dass das System sich erwartungsgemäß. Auf der anderen Seite findet sie es schlecht, dass die Datenbanken nach dem Abschluß der Migration nicht automatisch aktualisiert werden.


\subsection*{\textbf{Verbesserungsvorschläge}}
\label{sec:verbesserung}
Während des Interviews hat die Expertin vie Ideen vorgeschlagen, um das GuttenBase zu verbessern. Diese werden im Folgenden zusammengefasst:
\begin{itemize}
	\item \textbf{V1:} Menüpunkt \glqq Show Actions\grqq\, zu \glqq Show Migration Action\grqq\, umbenennen.
	\item \textbf{V2:} Menüpunkte \glqq Migrate Database\grqq\, und \glqq Show Migration Action\grqq\, gruppieren und unter dem Menüpunkt \glqq Database Tools\grqq\, bewegen. 
	\item \textbf{V3:} Schriftgröße von \glqq Source\grqq\, und \glqq Target\grqq\, bei der Datenbankverbindung vergrößern. 
	\item \textbf{V4:} Echtzeit Überprüfung von den Benutzereingaben während der Datenbankverbindung.
	\item \textbf{V5:} \glqq +\grqq\, Button bei der Konfigurationsübersicht umbenennen oder neben den Migrationsoperationen platzieren.
	\item \textbf{V6:} Ein automatisches Aktualisieren der Zieldatenbank ermöglichen oder einen entsprechenden Hinweis am Ende der Migration in der Fortschrittsübersict anzeigen, damit der Benutzer informiert wird.
\end{itemize}


%- Interview transkribieren
%- Kodierungsprozess: Kategorien erstellen
%- qualitative Inhaltsanalyse: strukturierende Inhaltsanalyse - \cite{ramsenthaler2013qualitative}
%- Kategorien: 
%	- Usability: 
%		- Fehlertoleranz
%		- Aufgabenangemessenheit
%		- Selbsbeschreibfähigkeit
%		- Zufriedenheit 
%	- Verbesserungsvorschläge
%	- Usability: 
%		- Erwartungskonformität: Hinweis Refresh hinzufügen oder refresh machen
%		- Selbstbeschreibungsfähigkeit: show migration actions umbenennen, plus button umbenennen / bewegen.
%		- Aufgabenangemessenheit: add actions button hinzufügen
%		- Fehlertoleranz: real time check general db verbindung 
%		- Konsistenz: Actions gruppieren, font ändern von src target, 
%	
%	- Gesamteindruck
%		- positives Feedback: Zufrieden
%		- Negatives Feedback: zusammenfassung aller - 
%		
%	- Verbesserungsvorschläge: Zusammenfassung vorschläge

\section{Anpassungen}
Nach der Evaluation wurden Anpassung vorgenomm, die schnell einsetzbar sind. diese werden im Folgenden aufgelisten:
\begin{itemize}
	\item \textbf{V1:} Menüpunkt \glqq Show Actions\grqq\, zu \glqq Show Migration Action\grqq\, umbenennen.
	\item \textbf{V2:} Menüpunkte \glqq Migrate Database\grqq\, und \glqq Show Migration Action\grqq\, gruppieren und unter dem Menüpunkt \glqq Database Tools\grqq\, bewegen. 
	\item \textbf{V3:} Schriftgröße von \glqq Source\grqq\, und \glqq Target\grqq\, bei der Datenbankverbindung vergrößern. 
	\item \textbf{V5:} \glqq +\grqq\, Button bei der Konfigurationsübersicht neben den Migrationsoperationen platzieren.
	\item \textbf{V6:} Hinweis am Ende der Migration in der Fortschrittsübersict anzeigen, damit der Benutzer informiert wird.
\end{itemize}
Es ist zu beachten, dass der Abschnitt \ref{sec:imp} entsprechend der neuen Anpassungen aktualisiert wurde.
\chapter{Fazit und Ausblick}
%Das zu Beginn der Bachelorarbeit gesetzte Ziel wurde mithilfe des GuttenBase Plugin erreicht. \\
%Mit dem GuttenBase Plugin lassen sich Datenbanken durch wenige Klicke migrieren. Während des Migrationsprozesses können Konfigurationsschritte hibzugefügt werden wie das Umbenennen der Quell-Tabellen bzw. Spalten sowie das Ausschließen bestimmter Tabellen bzw. Spalten und das Ändern von Spalten-Datentypen. Diese können vor- oder in dem  Migrationsprozess erstellt werden und werden gespeichert, um sie bei der nächsten Migration wiederverwenden zu können.\\
%Das GuttenBase Plugin läuft basierend auf der GuttenBase Bibliothek und interagiert mit dem Database Plugin von IntelliJ.\\
%\\ \\
\label{sec:fazit}
\section{Fazit}
Nach einer Einführung in die Grundlagen der Datenbank Migration, Guttenbase und die IntelliJ Plugin Entwicklung, wurde eine Anforderungsanalyse derchgeführt. basierend darauf wurde das GuttenBase Plugin entworfen, implementiert und evaluiert. \\
Mit dem GuttenBase Plugin lassen sich Datenbanken durch wenige Klicke migrieren. Während des Migrationsprozesses können Migrationsoperationen hibzugefügt werden wie das Umbenennen der Quell-Tabellen bzw. Spalten sowie das Ausschließen bestimmter Tabellen bzw. Spalten und das Ändern von Spalten-Datentypen. Diese können vor- oder in dem  Migrationsprozess erstellt werden und werden gespeichert, um sie bei der nächsten Migration wiederverwenden zu können.\\
Das GuttenBase Plugin läuft basierend auf der GuttenBase Bibliothek und interagiert mit dem Database Plugin von IntelliJ.\\
Mit den oben genannten Funktionalitäten erreicht das GuttenBase Plugin das zu Beginn der Bachelorarbeit gesetztes Ziel.
\section{Ausblick}
Die in dieser Bachelorarbeit implementierten Funktionalitäten stellen eine Grundlage dar, die sich beliebig ausbauen lässt. Zum Einen können die fehlenden Verbesserngsvorschläge aus dem Abschnitt \ref{sec:verbesserung} umgesetzt werden. Zum Anderen kann das GuttenBase Plugin um weitere Migrationsoperationen erweitert werden. 

Es spricht im Grunde nichts dagegen, das GuttenBase Plugin auf andere Entwicklungsumgebungen wie Eclipse oder NetBeans zu übertragen. Da einige Implementierungen von der IntelliJ API abhängig sind, müsste jede Funktionalität entsprechend der Schnittstelle der jeweiligen Entwicklungsumgebung implementiert werden. Eine Voraussetzung für eine solche Übertragung ist die Unterstützung von Datenbanken. Ansonsten müssen zusätzliche Informationen über die Quell- und Ziel-Datenbank erforderlich.

Es könnte außerdem vorkommen, dass einige Bugs während der Nutzung des Plugins auftreten. Diese können natürlich durch regelmäßige Wartung gelöst werden.



\printbibliography[heading=bibintoc,title={Literatur}]
%\printglossary[style=thesisacr,type=acronym,title=\abbreviationsname]
%\printglossary[style=thesis,type=main,title=\glossaryname]
\end{document}
