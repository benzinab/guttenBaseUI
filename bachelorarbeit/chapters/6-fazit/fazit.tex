\chapter{Fazit und Ausblick}
%Das zu Beginn der Bachelorarbeit gesetzte Ziel wurde mithilfe des GuttenBase Plugin erreicht. \\
%Mit dem GuttenBase Plugin lassen sich Datenbanken durch wenige Klicke migrieren. Während des Migrationsprozesses können Konfigurationsschritte hibzugefügt werden wie das Umbenennen der Quell-Tabellen bzw. Spalten sowie das Ausschließen bestimmter Tabellen bzw. Spalten und das Ändern von Spalten-Datentypen. Diese können vor- oder in dem  Migrationsprozess erstellt werden und werden gespeichert, um sie bei der nächsten Migration wiederverwenden zu können.\\
%Das GuttenBase Plugin läuft basierend auf der GuttenBase Bibliothek und interagiert mit dem Database Plugin von IntelliJ.\\
%\\ \\
\label{sec:fazit}
\section{Fazit}

Nach einer Einführung in die Grundlagen der Datenbankmigration, GuttenBase und in die IntelliJ-Plugin-Entwicklung wurde eine Anforderungsanalyse durchgeführt. Basierend darauf wurde das GuttenBase-Plugin entworfen, implementiert und evaluiert.\\
Mit diesem lassen sich Datenbanken mittels weniger Klicks migrieren. Während des Migrationsprozesses können Migrationsoperationen hinzugefügt werden, etwa das Umbenennen der Quelltabellen bzw. -spalten sowie das Ausschließen bestimmter Tabellen bzw. Spalten und das Ändern von Spalten-Datentypen. Diese können vor oder während des Migrationsprozesses erstellt werden und werden gespeichert, um bei der nächsten Migration abermals verwendet werden zu können. \\ Das GuttenBase-Plugin basiert auf der GuttenBase-Bibliothek und interagiert mit dem Database-Plugin von IntelliJ.\\
Mit den oben genannten Funktionalitäten erreicht das GuttenBase-Plugin das zu Beginn der Bachelorarbeit gesetzte Ziel

\section{Ausblick}

Die in dieser Bachelorarbeit implementierten Funktionalitäten stellen eine Grundlage dar, die sich beliebig ausbauen lässt. Zum einen können die noch fehlenden Optimierungen aus Abschnitt \ref{sec:verbesserung} umgesetzt werden. Zum anderen kann das GuttenBase-Plugin um weitere Migrationsoperationen erweitert werden.

Es spricht nichts dagegen, das GuttenBase-Plugin auf andere Entwicklungsumgebungen wie Eclipse oder NetBeans zu übertragen. Da einige Implementierungen vom IntelliJ-API abhängig sind, müsste jede Funktionalität entsprechend der Schnittstelle der jeweiligen Entwicklungsumgebung implementiert werden. Eine Voraussetzung für eine solche Übertragung ist die Unterstützung von Datenbanken. Ansonsten sind zusätzliche Informationen über die Quell- und Zieldatenbank erforderlich.

Es kann außerdem vorkommen, dass Bugs während der Nutzung des Plugins auftreten. Diese können mittels einer regelmäßigen Wartung behoben werden.