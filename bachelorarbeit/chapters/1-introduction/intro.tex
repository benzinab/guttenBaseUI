\chapter{Einleitung}


\section{Problemstellung}

%Problemstellung/Forschungsfrage: 
%Wie lässt sich ein Tool für DB Migration basierend auf GB entwickeln?
%Was sind die aktuellen Tools für Datenbank Migration?
%Wie lässt sich die GB Bibliothek optimieren?

Datenbank Migration ist seit Anbeginn des Informationszeitalter ein wichtiger Bestandteil der Informationsverarbeitung. Wie die Harware, Betriebssysteme und Programme, werden Datenbank auch häufig migriert. Der Auslöser könnte z. B. eine Umstrukturierung im Unternehmen sein. \\
Trotz der Relevanz der Datenbank Migration, ist die Entwicklung und die Forschung in diesem Bereich in den letzten Jahren sehr gering. Deswegen stellt sich die Frage, wie sich die Datenbank Migration optimieren lässt.\\

\section{Motivation und Zielsetzung}
Es gibt viele Tools zum Visualisieren oder Analysieren von  Datenbanken. Ebenfalls könnte man einige Programme für Datenbank Migration finden. Diese sind allerdings nicht flexibel genug bzw. decken nicht alle Anforderungen ab. Deswegen bietet sich die open Source Bibliothek GuttenBase von der Firma Akquinet AG an, die eine gewisse Flexibilität anbietet. Diese lässt sich jedoch stark optimieren, um eine eine schnellere, anpassbare und flexible Migration durchführen zu können.\\



%In Guttenbase muss man bei jeder Migration das ganze Mapping selber implementieren (überschreiben). \\
%Außerdem fehlt die Möglichkeit, eine Übersicht der Datenbank zu haben und Konfigurationsschritte hinzuzufügen, während man migriert.
%Aus diesen Gründen lässt sich die Nutzung der GuttenBase Bibliothek einschränken. Deswegen bietet sich die Möglichkeit, ein Tool zu realisieren, um die Nutzung der GuttenBase Bibliothek zu optimieren und eine flexible, einfache und konfigurierbare Datenbank Migration zu ermöglichen.

\section{Zielsetzung}
Die GuttenBase Bibliothek lässt sich durch unterschiedliche Weiterentwicklungen optimieren. Allerdings wird im Rahmend dieser Arbeit eine eigene Anwendungsoberfläche für Datenbank Migration basierend auf GuttenBase konzipiert, implementiert und anschließend evaluiert. 
\section{Aufbau der Arbeit}
Zu Beginn der Arbeit werden einige Grundbegriffe für Datenbank Migration erläutert. Außerdem werden die Eigenschaften der GuttenBase Bibliothek vorgestellt.\\
Zusätzlich werden aktuelle Tools für Datenbank Migration erwähnt und miteinander verglichen.\\
Der Hauptteil dieser Arbeit beschäftigt sich hauptsächlich mit der Umsetzung des Guttenbase Tools. Dabei wird zuerst die Umsetzungsform begründet und es werden die Schritte der Umsetzung genauer erläutert.\\
Im darauffolgenden Kapitel wird das Ergebnis kurz evaluiert und am anschließend gibt es eine Zusammenfassung sowie Ideen für Optimierungsmöglichkeiten.