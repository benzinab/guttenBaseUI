\chapter{Einleitung}

\section{Problemstellung und Motivation}

%Problemstellung/Forschungsfrage: 
%Wie lässt sich ein Tool für DB Migration basierend auf GB entwickeln?
%Was sind die aktuellen Tools für Datenbank Migration?
%Wie lässt sich die GB Bibliothek optimieren?

%\section{Problemstellung }
IT-Migration ist seit Anbeginn des Informationszeitalter ein wichtiger Bestandteil der Informationsverarbeitung \cite{wachter2015systemkonsolidierung}. Wie die Harware, Betriebssysteme und Programme, werden Datenbanken auch häufig migriert. Der Grund dafür könnte z. B. eine Änderung der Unternehmensrichtlinien sein. \\ 
Bei einer Datenbank Migration werden Daten von einer Quell-Datenbank zu einer Ziel-Datenbank verschoben. \\
%Trotz der Relevanz der Datenbank Migration, ist die Entwicklung und die Forschung in diesem Bereich in den letzten Jahren sehr gering. Aus diesem Grunde stellt sich die Frage, wie sich die Datenbank Migration optimieren lässt.\\
%\section{Motivation }
Es gibt viele Tools zum Visualisieren oder Analysieren von  Datenbanken. Ebenfalls gibt es einige Programme für Datenbank Migration. Dazu gehört die GuttenBase Bibliothek. Diese bietet durch das Hinzufügen von Migrationsoperationen eine gewisse Flexibilität während des Migrationsprozesses an. Um diese Migrationsoperationen am effizientesten auszunutzen und eine schnellere und anpassbare Migration durchzuführen, lässt sich GuttenBase stark optimieren. \\Diese Arbeit beschäftigt sich mit der Frage, wie sich die von der Firma Akquinet AG entwickelte Bibliothek optimieren lässt.
%In Guttenbase muss man bei jeder Migration das ganze Mapping selber implementieren (überschreiben). \\
%Außerdem fehlt die Möglichkeit, eine Übersicht der Datenbank zu haben und Konfigurationsschritte hinzuzufügen, während man migriert.
%Aus diesen Gründen lässt sich die Nutzung der GuttenBase Bibliothek einschränken. Deswegen bietet sich die Möglichkeit, ein Tool zu realisieren, um die Nutzung der GuttenBase Bibliothek zu optimieren und eine flexible, einfache und konfigurierbare Datenbank Migration zu ermöglichen.
\section{Zielsetzung}
Die GuttenBase Bibliothek lässt sich durch unterschiedliche Weiterentwicklungen optimieren. 
Im Rahmend dieser Arbeit sollte  eine eigene Anwendungsoberfläche (GuttenBase Plugin) für Datenbank Migration basierend auf GuttenBase konzipiert, implementiert und anschließend evaluiert werden. \\ \\
Das GuttenBase Plugin soll die wichtigsten Funktionalitäten von GuttenBase unterstützen. Diese werden bei der Anforderungsanalyse genauer erläutert (siehe Abschnitt \ref{anaylse}).\\
Um ein benutzerfreundliches System zu erzielen, ist es wichtig dass die zu entwickelnde Anwendungsoberfläche den Grundsätzen der Informationsdarstellung entsprechen. Diese wurden in der Norm DIN EN ISO 9241-112 vorgestellt und beinhalten folgende Grundsätze:
\begin{itemize}
	\item Entdeckbarkeit: 
	Informationen sollen bei der Darstellung erkennbar sein und als vorhanden wahrgenommen werden.
	
	\item Ablenkungsfreiheit: 
	Erforderliche Informationen sollen wahrgenommen werden, ohne Störung von weiteren dargestellten Informationen.
	
	\item Unterscheidbarkeit: 
	Elemente oder Gruppen von elementen sollen voneinander unterschieden werden können. Die Darstellung sollte die Unterscheidung bzw. Zuordnung von Elementen und Gruppen unterstützen.
	
	\item Eindeutige Interpretierbarkeit:
	Informationen sollen verstanden werden, wie es vorgesehen ist.
	
	\item Kompaktheit:
	Nur notwendige Informationen sollen dargestellt werden.
	
	\item Konsistenz:
	Informationen mit ähnlicher Absicht söllen ähnlich dargestellt werden und Informationen mit unterschiedlicher Absicht sollen in unterschiedlicher Form dargestellt werden.
	
\end{itemize}
Die genannten Grundsätze sollen im Zusammenhang mit den Gründsätzen für die Benutzer-System-Interaktion („Dialogprinzipien“) angewendet werden. Diese beinhalten, Nach der Norm DIN EN ISO 9241-11, folgende Grundsätze:
\begin{itemize}
	\item Aufgabenangemessenheit:
	Schritte sollen nicht überflüssig sein und keine irreführende Informationen beinhalten.
	
	\item Selbstbeschreibungsfähigkeit:
	Es sollen nur genau die Informationen dargestellt werden, die für einen bestimmten Schritt erforderlich sind.
	
	\item Erwartungskonformität:
	Das System verhält sich nach Durchführung einer bestimmten Aufgabe wie erwartet.
	
	\item Lernförderlichkeit:
	der Benutzer kann den entsprechenden Schritt durchführen, eine ein Vorwissen bzw. eine Schulung zu haben.
	
	\item Steuerbarkeit:
	Der Benutzer kann konsequent und ohne Umwege in Richtungen gehen, die für die zu erledigende Aufgabe erforderlich sind.
	
	\item Fehlertoleranz:
	Das System soll den Benutzer vor Fehlern schützen, und wenn Fehler gemacht werden, sollen diese mit minimalen Aufwand behoben werden können.

	\item Individualisierbarkeit:
	Der Benutzer kann Anwendungsoberfläche durch individuelle Voreinstellugen anpassen.
	
\end{itemize}
Die oben genannten Grundsätze stellen sicher, dass das GuttenBase Plugin effektiv, effizient und zufriedenstellend ist. Diese sind die drei Ziele der Gebrauchstauglichkeit (Usability).

\section{Aufbau der Arbeit}
Zu Beginn der Arbeit werden einige Grundbegriffe für Datenbank Migration erläutert. Außerdem werden die Eigenschaften der GuttenBase Bibliothek vorgestellt.\\
Zusätzlich werden aktuelle Tools für Datenbank Migration erwähnt.\\
Der Hauptteil dieser Arbeit beschäftigt sich hauptsächlich mit der Umsetzung des Guttenbase Plugins. Dabei wird zuerst eine Anforderungsanlyse durchgeführt, um den Soll-Zustand zu definieren. Um die technische Machbarkeit zu prüfen und Zeit bei der Entwicklung zu sparen, werden bei der Analyse einige GUI-Prototypen erstell. Somit wird am Anfang der Umsetzung klar sein, wie die zu entwickelnde Anwendungsoberfläche die Funktionalitäten von Guttenbase unterstützen würde. Außerdem wird die Umsetzungsform begründet. \\
Die Software Architektur erfolgt im darauffolgenden Abschnitt. Diese wird basierend auf den Siemens Blickwinkel erstellt. Zunächst werden die verwendeten Technologien sowie die implementierten Features vorgestellt.\\
Im darauffolgenden Kapitel wird das Ergebnis kurz evaluiert.
Um das Ergebnis zu evaluieren, wird ein Experten-Interview durchgeführt.
anschließend gibt es eine Zusammenfassung sowie Ideen für Optimierungsmöglichkeiten.