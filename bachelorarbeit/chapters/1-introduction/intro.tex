\chapter{Einleitung}

\section{Problemstellung und Motivation}

%Problemstellung/Forschungsfrage: 
%Wie lässt sich ein Tool für DB Migration basierend auf GB entwickeln?
%Was sind die aktuellen Tools für Datenbankmigration?
%Wie lässt sich die GB Bibliothek optimieren?

%\section{Problemstellung }
Die IT-Migration ist seit Anbeginn des Informationszeitalter ein zentraler Bestandteil der Informationsverarbeitung \cite{wachter2015systemkonsolidierung}. Ebenso wie Harware, Betriebssysteme sowie Programme, werden auch Datenbanken häufig migriert. Bei einer Datenbankmigration werden Daten von einer Quelldatenbank in eine Zieldatenbank verschoben. Die Notwendigkeit einer Datenbankmigration besteht darin, ein vorhandenes System auf ein modernes auszutauschen, das den aktuellen Anforderungen entspricht \cite{housel1974architecture}.

%Trotz der Relevanz der Datenbankmigration, ist die Entwicklung und die Forschung in diesem Bereich in den letzten Jahren sehr gering. Aus diesem Grunde stellt sich die Frage, wie sich die Datenbankmigration optimieren lässt.\\
%\section{Motivation }
Es gibt zahlreiche Tools zum Visualisieren oder Analysieren von  Datenbanken (Tableau \cite{datig2018telling}, SAP Crystal Reports \cite{duttaroy2016sap} etc.). Zudem gibt es einige Programme für die Datenbankmigration \cite{horstmann2005migration}. Dazu gehört die GuttenBase-Bibliothek. Diese bietet durch die Möglichkeit der Hinzufügens von Migrationsoperationen eine gewisse Flexibilität während des Migrationsprozesses. Um diese Migrationsoperationen effizient auszunutzen und eine raschere und anpassbare Migration durchzuführen, lässt sich GuttenBase optimieren. \\ In dieser Arbeit wird der Frage nachgegangen, Wie ein Plugin anhand der GuttenBase-Bibliothek entwickelt werden kann, das eine optimale Anwendungsoberfläche für die Datenbankmigration bietet.
%In Guttenbase muss man bei jeder Migration das ganze Mapping selber implementieren (überschreiben). \\
%Außerdem fehlt die Möglichkeit, eine Übersicht der Datenbank zu haben und Konfigurationsschritte hinzuzufügen, während man migriert.
%Aus diesen Gründen lässt sich die Nutzung der GuttenBase Bibliothek einschränken. Deswegen bietet sich die Möglichkeit, ein Tool zu realisieren, um die Nutzung der GuttenBase Bibliothek zu optimieren und eine flexible, einfache und konfigurierbare Datenbankmigration zu ermöglichen.
\section{Zielsetzung}
\label{sec:ziel}
Die GuttenBase-Bibliothek lässt sich mithilfe unterschiedlicher Weiterentwicklungen optimieren. 
Im Rahmen dieser Arbeit soll eine Anwendungsoberfläche (GuttenBase-Plugin) für die Datenbankmigration basierend auf GuttenBase konzipiert, implementiert und anschließend evaluiert werden. \\ \\
Das GuttenBase-Plugin soll die bedeutendsten Funktionalitäten von GuttenBase unterstützen. Diese werden in der Anforderungsanalyse erläutert (siehe Kapitel \ref{sec:analyse}).\\
Um ein benutzerfreundliches System zu erzielen, ist es ausschlaggebend, dass die zu entwickelnde Anwendungsoberfläche den Grundsätzen der Informationsdarstellung entspricht. Diese sind in der Norm DIN EN ISO 9241-112 zu finden und beinhalten die folgenden Grundsätze:
\begin{itemize}
	\item Entdeckbarkeit: 
	Die Informationen sollen in der Darstellung erkennbar sein und als vorhanden wahrgenommen werden können.
	
	\item Ablenkungsfreiheit: 
	Die erforderlichen Informationen sollen ohne eine durch weitere Informationen verursachte Störung wahrgenommen werden.
	
	\item Unterscheidbarkeit: 
	Elemente oder Gruppen von elementen sollen voneinander unterschieden werden können. Die Darstellung sollte die Unterscheidung bzw. Zuordnung von Elementen und Gruppen unterstützen.
	
	\item Eindeutige Interpretierbarkeit:
	Die Informationen sollen auf eine Weise verstanden werden, wie es vorgesehen ist.
	
	\item Kompaktheit:
	Nur notwendige Informationen sollen dargestellt werden.
	
	\item Konsistenz:
	Ähnliche Absichten sollen ähnlich dargestellt werden und unterschiedliche Absichten sollen in unterschiedlicher Form dargestellt werden.
	
\end{itemize}
Die genannten Grundsätze sollen im Zusammenhang mit den Grundsätzen der Benutzer-System-Interaktion („Dialogprinzipien“) angewendet werden. Diese beinhalten nach der Norm DIN EN ISO 9241-11 die folgenden Grundsätze:
\begin{itemize}
	\item Aufgabenangemessenheit:
	Die Schritte sollen nicht überflüssig sein und keine irreführende Informationen beinhalten.
	
	\item Selbstbeschreibungsfähigkeit:
	Es sollen lediglich jene Informationen dargestellt werden, die für einen bestimmten Schritt erforderlich sind.
	
	\item Erwartungskonformität:
	Das System verhält sich nach der Durchführung einer bestimmten Aufgabe wie erwartet.
	
	\item Lernförderlichkeit:
	Der Benutzer kann den entsprechenden Schritt durchführen, ohne Vorwissen zu haben bzw. eine Schulung zu besuchen.
	
	\item Steuerbarkeit:
	Der Benutzer kann konsequent und ohne Umwege in Richtungen gehen, die für die zu erledigende Aufgabe erforderlich sind.
	
	\item Fehlertoleranz:
	Das System soll den Benutzer vor Fehlern schützen. Wenn Fehler gemacht werden, sollen diese mit minimalem Aufwand behoben werden können.

	\item Individualisierbarkeit:
	Der Benutzer kann die Anwendungsoberfläche mittels individueller Voreinstellungen anpassen.
	
\end{itemize}
Mit den oben genannten Grundsätzen wird sichergestellt, dass das GuttenBase-Plugin effektiv, effizient und zufriedenstellend ist. Diese drei Merkmale entsprechen den drei Zielen der Gebrauchstauglichkeit (Usability).

\section{Aufbau der Arbeit}
Zu Beginn der Arbeit werden einige Grundbegriffe der Datenbankmigration erläutert. Außerdem werden die Eigenschaften der GuttenBase-Bibliothek aktuelle Tools für die Datenbankmigration erläutert (Kapitel \ref{sect:grundlagen}).\\
Der Hauptteil dieser Arbeit hat die Umsetzung des Guttenbase-Plugins zum Thema. Dabei wird zuerst eine Anforderungsanalyse durchgeführt, um den Sollzustand zu definieren (Kapitel \ref{sec:analyse}). Um die technische Machbarkeit zu prüfen und Zeit bei der Entwicklung zu sparen, werden bei der Analyse einige GUI-Prototypen erstellt. Somit wird bereits zu Beginn der Umsetzung klar sein, wie die zu entwickelnde Anwendungsoberfläche die Funktionalitäten von GuttenBase unterstützen würde. \\
Die Softwarearchitektur erfolgt im darauffolgenden Kapitel (Kapitel \ref{sec:kzimp}). Diese wird basierend auf dem Siemens-Blickwinkel erstellt. Zunächst werden die verwendeten Technologien sowie das Ergebnis vorgestellt.\\
Im darauffolgenden Kapitel (\ref{sec:evaluation}) wird das Ergebnis evaluiert. Dabei wird ein Experteninterview durchgeführt.\\
Anschließend folgen eine Zusammenfassung sowie Ideen für Optimierungsmöglichkeiten (Kapitel \ref{sec:fazit}).