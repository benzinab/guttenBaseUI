\chapter{Grundlagen}
Dieses Kapitel liefert einen allgemeinen Einblick einige Grundaspekte der Datenbank Migration sowie der GuttenBase Bibliothek. Außerdem werden verwandte Arbeiten vorgestellt. 
\section{Datenbanken}
%- allgemeine Def
%- Arten von Datenabnken
%https://books.google.de/books?hl=de&lr=&id=_3XVAwAAQBAJ&oi=fnd&pg=PA14&dq=Datenbank+Grundlagen+&ots=ntu4qhQNQD&sig=LpAgd_s5f04ulLTfOYe6lEt1_Zc#v=onepage&q=Datenbank%20Grundlagen&f=false
Datenbanken spielen seit der Neuerung des IT-Zeitalter eine wichtige Rolle in dem elektronischen Datenmanagement.\\
Eine Datenbank ist eine geordnete, selstbeschreibende Sammlung von Daten, die miteinander in Beziehung stehen.
Vielmehr ist eine Datenbank ein verteiltes, integriertes Computersystem, das Nutzdaten und Metadaten enthält. Nutzdaten sind dabei die Daten, die Benutzer in der Datenbank anlegen und aus denen die Informationen gewonnen werden. Metadaten werden of auch als Daten über Daten bezeichnet und helfen, die Nutzdaten der Datenbank zu strukturieren. 
	
	
	
%\section{Datenbank Management System (DBMS)}
%- Def
%- Beispiele
Damit Datenbanken auf einem Computer verwaltet werden können, werden Datenbankmanagement Systeme (DBMS) benötigt. Diese sind leistungsfähige Programme für die flexible Speicherung und Abfrage strukturierter Daten. \\
Außerdem hilft ein DBMS bei der Organisation und Integrität von Daten und regelt den Zugruff auf Datengruppen. \\
Ein DBMS kann aus einem einzelenen Programm bestehen. Dies ist z. B. bei einem Desktop-DBMS zu sehen. Es kann jedoch aus verschiedenen Programmen bestehen, die zusammenarbeiten und die Funktion des DBMS bereitstellen. Dies ist z. B. bei den servergestützen Datenbanksystemen der Fall.\\
Um eine Datenbank Anwendung zu implementieren, sollte auf das Datenbankmodell geachtet werden. Dies stellt die Daten einer Datenbank und deren Beziehungen abstrakt dar. Meistens wird ein relationales Datenbankmodell eingesetzt. Dies hat, im Gegensatz zu den anderen Datenbankmodellen, keine strukturelle Abhängigkeit und versteckt die physikalische Komplexität der Datenbank komplett vor den Anwendern.\\
Es stehen zahlreiche Datenbankmanagementsysteme zur Verfügung. Folgendes befinden sich einige der gängigsten DBMS:
\begin{itemize}
	\item Microsoft SQL Server
	\item MS-Access
	\item MySQL
	\item PostgreSQL
	\item HSQLDB
	\item H2 Derby
	\item Oracle
	\item DB2
	\item Sybase
\end{itemize}
Um ein geeignetes DBMS auszuwählen, gibt es viele Kriterien wie die Ausführungszeit, CPU- und Speicher Nutzung. Der Artikel von Youssif Bassil, A Comparative Study on the Performance of the Top DBMS Systems, im Jahr 2011 vergleicht einige Datenbankmanagementsysteme anhand der genannten Kriterien.
%https://arxiv.org/pdf/1205.2889.pdf
\section{Datenbank Migration}

%- Was allgemeines (1 Satzt)
%- Was ist Datenbank Migration
%- Wieso wird Datenbank Migration benötigt?
%- Welche Arten von Datenbank Migrationen gibt es?



Datenbank Migration wird immer mehr von Unternehmen bzw. Organisationen gebraucht. 

Die Migration von Datenbanken dient zum Verschieben der Daten von der Quell-Datenbank zur Ziel-Datenbank einschließlich die Schemaübersetzung und Datentransformation.


Mögliche Gründe für eine Dantenbank Migration sind:
\begin{itemize}
	\item Upgrade auf eine neue Software oder Hardware
	\item Änderung der Unternehmensrichtlinien
	\item Investition in IT-Diienstleistungen
	\item Integration von Datenquelle in ein System
	\item Zusammenführen mehrerer Datenbanken in einer Datenbank für eine einheitliche Datenansicht.
	\item Wartung des existierenden Systems ist schwer oder nicht möglich.
\end{itemize}

Außerdem gibt es unterschiedliche Strategien für Datenbank Migration. Diese können in drei Kategorien unterteilt werden:

\begin{enumerate}
	\item Migration durch objekt orientierte Schnittstellen: \\
	Bei dieser Strategie werden Daten in form von Objekten bzw. XML Dateien verarbeitet. Dafür wird ein bidirektionales Mapping benötigt, ojektbasierte Schemas in Datenbank Schemas zu übersetzten.
	\item Datenbank Integration: \\
	Hier wird die Quell-Datenbank mit der Ziel-Datenbank verbunden, wodurch der Eindruck entsteht, als ob alle Daten in einer einzigen Datenbank gespeichert sind.	
	\item Datenbank Migration: \\
	Die Quell-Datenbank wird in die Ziel-Datenbank kopiert. Dabei werden Schemas in ein Zielschema semantisch übersetzt werden. Darauf basierend werden die enthaltenen Daten konvertiert.
\end{enumerate}



\section{Verwandte Arbeiten}
Eines der Hauptprobleme in der Softwareindustrie besteht darin, eine hochwertige Datenverwaltung sicherzustellen. Dies ist auch der Fall bei einer Datenbank Migration, wobei die mit dem Migrationsworkflow verbundenen Aufgaben vielfältig und kompliziert sind. Das manuelle Ausführen dieser Aufgaben erfordert viel Zeit und ein sehr erfahrenes Team. Um Zeit und Kosten bei der Migration zu sparen bieten sich zahlreiche Tools bzw. Prototypen für Datenbank Migration (DBMT für Databbase Migration Tool).\\
In der folgenden Tabelle werden einige dieser Tools dargestellt: 
%TODO: add table and table ref
\begin{center}
	\begin{tabular}{ |p{2cm}|p{3cm}|p{4cm}|p{4cm}| }
		\hline
		\textbf{Name}} & \textbf{Unterstützte DBMS} & \textbf{Vorteile} & \textbf{Nachteile} \\
		\hline
		 SQuirrel DBCopy Plugin & 
		 Axion,
		 DB2,
		 DaffodilDB (One\$DB),
		 Derby(Cloudscape),
		 Firebird,
		 Frontbase,
		 HyperSonic(HSQLDB),
		 H2,
		 Ingres II,
		 MaxDB (SAP),
		 McKoi,
		 MySQL,
		 Microsoft SQL Server,
		 Oracle,
		 Pointbase,
		 PostgreSQL,
		 Progress,
		 Sybase ASE,
		 TimesTen  & & \\
		 \hline
		 OSDM Toolkit & 
		 Quell-DBMS: Oracle, SyBase, Informix, DB2, MS Access, MS SQL, Ziel-DBMS: PostgreSQL, MySQL & 
		 Alle  Betriebssysteme werden unterstützt, Open Source &  \\
		 \hline
		 & & & \\
		 \hline
		 & & &  \\
		 \hline
		 & & &  \\
		\hline
	\end{tabular}
\end{center}

%TODO: Vergeich als Tabelle.

\section{GuttenBase}
Viele Software Unternehmen haben sich dafür entschieden, ein eigenes Tool für Datenbankmigration zu entwickeln. Dies ist der Fall bei der Firma Akquinet AG, wo die Open Source Bibliothek "GuttenBase" in 2012 entwickelt wurde. Da GuttenBase open source ist, wurde sie in weiteren Schritten weiterentwickelt und um zusätzliche Funktionen erweitert.\\
Anderes als die oben genannten Tools, bietet die GuttenBase Bibliothek eine gewisse Flexibilität bei der Migration. Diese kann durch das Programmieren und Überschreiben der Mapping Klassen spezifiziert werden, damit die Migration passend zu dem aktuellen Stand der Daten läuft.\\
Dieser Ansatz erlaubt Entwicklern, eine volle Kontrolle über den Migrationsprozess zu haben.\\
Für die Migration einer Datenbank ist häufig eine benutzerdefinierte Lösung erforderlich. Beispilsweise z. B. das Unbenennen von Tabellen bzw. Spalten in der Zieldatenbank, das Umwandeln von Spaltentypen, das Ausschließen von bestimmten Tabellen bzw. Spalten.
In diesem Fall können Konfigurationshinweise vor der Migration hinzugefügt werden. Standardmäßig wird eine Standardimplementierung der Hinweise nach dem Verbinden der Datenbanken hinzugefügt. Dise können jedoch von dem Nutzer überschrieben werden. \\
Folgendes ist eine Übersicht über die aktuelle Hinweise von GuttenBase: \\
%TODO Tabelle hinzufügen!

%- Allgemeiner Satz
%- Was ist GuttenBase
%- Wann wurde sie eingeführt?
%- Von wem ist sie entwickelt?
%- Warum soll man GuttenBase benutzten
%- Wie kann man GuttenBase benutzen?
%- [Was kann man in GuttenBase optimieren?]


