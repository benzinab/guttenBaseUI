\chapter{Evaluation}
%- Allgemeines: Usability, Kriterien,... 
\label{sec:evaluation}
In den folgenden Abschnitten werden die erstellten Plugins bewertet. Zunächst werden die Evaluationsmethodik sowie der Ablauf der Evaluation beschrieben und anschließend werden die Ergebnisse vorgestellt.

\section{Evaluationsmethodik}

Um einen Gesamteindruck von der Gebrauchstauglichkeit (Usability) zu erhalten und etwaige Anpassungsmöglichkeiten zu identifizieren, wurde das entwickelte Tool in Form eines qualitativen Experteninterviews evaluiert.

Es werden in der Regel mindestens fünf Testpersonen benötigt, um den Großteil der in einem System auftretenden Probleme (85 \%) zu identifizieren [NL93]. Allerdings wurde im Rahmen dieser Bachelorarbeit aufgrund beschränkter zeitlicher Ressourcen lediglich ein Experteninterview durchgeführt.

die bedeutsamste Voraussetzung bei der Auswahl der Expertin war die Erfahrung mit Benutzerschnittstellen und der User-Experience. Außerdem waren Kenntnisse im Bereich der Datenbankverwaltung bzw. der Datenbankmigration von Vorteil.

\section{Durchführung}

Üblicherweise werden Experteninterviews in Form von Leitfadeninterviews durchgeführt \cite{mayring1994qualitative}. Dies ermöglicht die Erhebung von einzelnen, bestimmbaren Informationen sowie die Verfolgung konkreter Informationsziele. Daher wurde diese Befragungsart für die Evaluation gewählt. Bei dieser Befragungsmethode werden häufig offene Fragen gestellt, um mehr Freiraum für Antworten und möglichst viele Informationen zu gewinnen.

Das Experteninterview fand online über ,Teams‘\footnote{https://teams.microsoft.com/} statt. Der Hauptgrund dafür war die aktuelle Coronapandemie. Außerdem können Onlineinterviews flexibler als physische Interviews vereinbart werden.

Während des Experteninterviews wurde das GuttenBase-Plugin über einen geteilten Bildschirm angezeigt.
Dabei wurden die zentralen Funktionalitäten vorgestellt. Zunächst wurden Fragen aus einem Leitfaden gestellt, der vorab erstellt worden war. Der Leitfaden beinhaltet hauptsächlich Fragen über die Grundsätze der Informationsdarstellung und der Benutzer-System-Interaktion, die in Abschnitt \ref{sec:ziel} erläutert wurden. Außerdem wurden Fragen hinsichtlich der Verbesserungsmöglichkeiten gestellt.



%
%Struktur:
%- was : expert interview
%- Warum: Gesamteindruck + Verbesserungsvorschläge -> schnelle Anpassungen
%- nur 1 expert intervieww (keine zeit)
%
%
%
%
%
%\cite{franziska18interview}
%- : experteninterview (nur 1, da keine Zeit), auchh wenn 5 mindestens benötigt werden \cite{nielsen1993mathematical}
%- Ziel: Machbarkeit + schnelle Anpassungen 
%%- Wie: Proof of Concept??
%- Befragungsmethode: Leitfadeninterview (ziel: Informationsziel verfolgen) - Semistrukturiertes Interview - offene Fragen
%- Auswahl Expertin: Erfahrung in UX/UI + Kenntnisse in DBMS.
%- Inhalt:
%	- Ersteindruck
%	- Usability Grundsätze
%	- Vrbesserungsvorschläge
%- Ablauf: Online Meetin + Screensharing + Zugriff auf Screen
%- Interview Durchführung

\section{Ergebnisse}
Um das Interview auswerten zu können, wurden die Gespräche nach der Durchführung transkribiert. Zunächst wurde das transkribierte Interview  gemäß der qualitativen Inhaltsanalyse kodiert \cite{mayring1994qualitative}. Dabei wurde das Transkript gelesen und den Textfragmenten wurden bestimmte Kodes zugeordnet. Nach der Transkription erfolgte eine Kategorienbildung. Dabei wurden die folgenden Kategorien identifiziert, mit denen relevante Kodes zusammengefasst werden können:
\begin{itemize}
	\item Eindruck
	\item Usability
	\item Verbesserungsvorschläge
\end{itemize}

Im Folgenden werden die Ergebnisse des Interviews entsprechend den genannten Kategorien zusammen gefasst.

\subsection*{\textbf{Eindruck}}
Den Gesamteindruck der Anwendungsoberfläche bezeichnete die Expertin als zufriedenstellend und positiv. Die Benutzeroberfläche wirke mit der IntelliJ-Entwicklungsumgebung insgesamt intuitiv und einheitlich. Außerdem empfand die Expertin die Datenbankmigration in mehreren Schritten als mühelos.
 
\subsection*{\textbf{Usability}}
%toto

Es wurden von der Teilnehmerin mehrere Aspekte der Gebrauchstauglichkeit bewertet. Beispielsweise bewertete sie die \textbf{Fehlertoleranz} als positiv. Sie gab an, dass an beinahe allen Stellen eine Fehlermeldung angezeigt werde, die die Ursache des Fehlers erklärt. Dies ist bei der Datenbankverbindung zu sehen, wenn die Zugangsdaten nicht korrekt sind. Dabei wird die Fehlermeldung allerdings spät angezeigt (erst nach dem Klick auf den ,Next‘-Button), was die Expertin als verbesserungswürdig bezeichnete.

Die \textbf{Selbstbeschreibfähigkeit} bezeichnete die Teilnehmerin als verbesserungsfähig. Dies liege hauptsächlich an der Platzierung des ,Plus‘-Button in der Konfigurationsübersicht des Plugins. Die Teilnehmerin konnte die Funktion des Buttons nicht erkennen. Außerdem gab sie an, dass die Bezeichnungen der Menüpunkte für das Öffnen der Benutzeroberfläche nicht selbstbeschreibend seien.

Die Teilnehmerin bewertete die \textbf{Aufgabenangemessenheit} generell positiv. Allerdings fand sie das Hinzufügen von Migrationsoperationen während der Migration bedeutend. Diese Funktion wird aktuell allerdings nicht unterstützt.

Zur \textbf{Konsistenz} äußerte sich die Expertin insgesamt positiv. Sie empfand das Trennen von den Quell- und Zieldatenbanken bei der Datenbankverbindung als ausreichend. Allerdings meinte sie, dass die Schriftgröße der ,Source‘- und ‚Target‘-Überschriften zu klein sei. Laut ihrer Meinung sei außerdem die Gruppierung der Menüpunkte beim Öffnen der Benutzeroberfläche verbesserungswürdig.

Die Teilnehmerin bezeichnete die \textbf{Erwartungskonformität} als ,relativ gut‘. Sie gab an, dass sich das System erwartungsgemäß funktioniert. Auf der anderen Seite empfand sie es als unzureichend, dass die Datenbanken nach dem Abschluss der Migration nicht automatisch aktualisiert werden.


\subsection*{\textbf{Verbesserungsvorschläge}}
\label{sec:verbesserung}
Während des Interviews nannte die Expertin zahlreiche Ideen, um das GuttenBase-Plugin zu optimieren. Diese Vorschläge werden im Folgenden zusammengefasst:
\begin{itemize}
	\item \textbf{V1:} Menüpunkt ‚Show Actions‘ zu ‚Show Migration Action‘ umbenennen,
	\item \textbf{V2:} Menüpunkte ‚Migrate Database‘ und ‚Show Migration Action‘ gruppieren und unter den Menüpunkt ‚Database Tools‘ bewegen,
	\item \textbf{V3:} Schriftgröße von ‚Source‘ und ‚Target‘ bei der Datenbankverbindung vergrößern,
	\item \textbf{V4:} Echtzeitüberprüfung der Benutzereingaben während der Datenbankverbindung,
	\item \textbf{V5:} ‚+‘-Button in der Konfigurationsübersicht umbenennen oder neben den Migrationsoperationen platzieren,
	\item \textbf{V6:} ein automatisches Aktualisieren der Zieldatenbank ermöglichen oder einen entsprechenden Hinweis am Ende der Migration in der Fortschrittsübersicht anzeigen, damit der Benutzer informiert wird.
\end{itemize}


%- Interview transkribieren
%- Kodierungsprozess: Kategorien erstellen
%- qualitative Inhaltsanalyse: strukturierende Inhaltsanalyse - \cite{ramsenthaler2013qualitative}
%- Kategorien: 
%	- Usability: 
%		- Fehlertoleranz
%		- Aufgabenangemessenheit
%		- Selbsbeschreibfähigkeit
%		- Zufriedenheit 
%	- Verbesserungsvorschläge
%	- Usability: 
%		- Erwartungskonformität: Hinweis Refresh hinzufügen oder refresh machen
%		- Selbstbeschreibungsfähigkeit: show migration actions umbenennen, plus button umbenennen / bewegen.
%		- Aufgabenangemessenheit: add actions button hinzufügen
%		- Fehlertoleranz: real time check general db verbindung 
%		- Konsistenz: Actions gruppieren, font ändern von src target, 
%	
%	- Gesamteindruck
%		- positives Feedback: Zufrieden
%		- Negatives Feedback: zusammenfassung aller - 
%		
%	- Verbesserungsvorschläge: Zusammenfassung vorschläge

\section{Anpassungen}
Nach der Evaluation wurden Anpassungen vorgenommen, die rasch umsetzbar waren. Diese werden im Folgenden aufgelistet:
\begin{itemize}
	\item \textbf{V1:} Menüpunkt ‚Show Actions‘ zu ‚Show Migration Action‘ umbenannt,
	\item \textbf{V2:} Menüpunkte ‚Migrate Database‘ und ‚Show Migration Action‘ gruppieren und unter den Menüpunkt ‚Database Tools‘ bewegen,
	\item \textbf{V3:} Schriftgröße von ‚Source‘ und ‚Target‘ bei der Datenbankverbindung vergrößern, 
	\item \textbf{V5:} ‚+‘-Button in der Konfigurationsübersicht neben den Migrationsoperationen platzieren,
	\item \textbf{V6:} Hinweis am Ende der Migration in der Fortschrittsübersicht anzeigen, damit der Benutzer informiert wird.
\end{itemize}
Es ist zu beachten, dass Abschnitt \ref{sec:imp} entsprechend den neuen Anpassungen aktualisiert wurde.